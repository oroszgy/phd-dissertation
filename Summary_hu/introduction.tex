% !TeX spellcheck = hu_HU

A modern nyelvtechnológiai alkalmazások szerves részét képezik mindennapi életünknek.
% melyek segíthetik a számítógéppel vagy éppen a másik emberrel való kommunikációt.
Ilyen eszközök például azok, amik segítenek dokumentumaink helyesírásának vizsgálatában, idegen nyelvű források fordításában és megértésében, illetve az interneten található információk visszakeresésében.
A szövegfeldolgozás a nyelvtechnológia azon ága, amelynek célja digitalizált szövegek automatikus elemzése.
Az ilyen feladatokhoz szükséges nyelvi előfeldolgozást legtöbbször több lépésben oldhatjuk meg: először szó- és mondathatárok megállapítása szükséges, majd a szavak morfológiai elemzése és egyértelműsítése következhet, végül pedig a mondatok szintaktikai analízise, illetve a szöveg szemantikai értelmezése végezhető el.
A gyakorlatban az egyes részfeladatokat végző komponensek egymásra épülnek.
Bár ezen eszközök nem mindegyike érhető el minden nyelvre, két előfeldolgozó lépés mégis elengedhetetlen a szövegek magasabb szintű kezeléséhez. 
Először is a tokenek és a mondatok azonosítása létfontosságú, hiszen ezek az entitások az alapegységei a nyelvtechnológiai elemző rendszereknek.
Ezen túl, gyakorta szükséges még a szavak szófaji címkéinek és szótöveinek meghatározása is.

A legtöbb esetben az első részfeladat megoldottnak tekinthető, mivel a létező alkalmazások nagy pontossággal képesek szövegek részekre bontására.
Ennek ellenére, számos olyan nyelvterület létezik, amelyekre a jelenleg elérhető eszközök nem nyújtanak kielégítő eredményt.

%A szegmentálás problémáján kívül egy másik közismert feladat még a szófaji címkézés (PoS taggelés) is.
Az utóbbi húsz év során számos nagy pontosságú szófaji egyértelműsítő eszköz készült, melyek a legtöbb élő nyelvre elérhetőek.
A gyakorlatban azonban ezek legtöbbször adatvezérelt módszerekre épülnek, ezért teljesítményük nagyban függ a használt tanítóanyagtól.
Egy másik probléma, hogy a terület kutatásában az élenjáró eljárások elsődleges célja mindig is az angol nyelv elemzése volt, ezért a létrejött algoritmusok sokszor nem képesek kezelni a morfológiailag komplex nyelvek által okozott nehézségeket.
Így például az agglutináló nyelvek esetén a szavak szófajainak megállapítása nem elégséges (az angollal szemben).
Az elemzési lánc további komponensei a teljes morfoszintaktikai címke illetve a szótő ismeretét is igénylik.
Ezért megállapítható, hogy a mai nyelvtechnológia nem rendelkezik olyan algoritmusokkal, amelyek morfológiailag gazdag nyelvek esetén megfelelően működnének, továbbá kevés erőforrással rendelkező doménekre is hatékonyan alkalmazhatóak lennének.

Következésképpen, dolgozatunk célja kettős.
Először olyan morfológiai egyértelműsítő eljárásokat vizsgáltunk, melyek megfelelően kezelik az agglutináló nyelvek okozta problémákat, mindazonáltal egyszerű domén adaptációs feladatok esetén is alkalmazhatóak.
Másodsorban pedig olyan módszerekkel foglalkoztunk, melyek kevés erőforrással rendelkező domének elemzésére is alkalmasok.

Első lépésben azt vizsgáltuk, hogy \textbf{hogyan lehetséges létező szófaji címkéző eljárásokat a teljes morfológiai egyértelműsítés feladatára használni, úgy, hogy azok képesek legyenek kezelni az agglutináló nyelvek tipikus nehézségeit, továbbá alkalmazhatóak maradjanak egyszerűbb domén adaptációs feladatokra is}.
Így létrehoztunk egy magas pontossággal rendelkező lemmatizáló algoritmust, amire épülve kifejlesztettünk egy teljes egyértelműsítő rendszert (PurePos).
Ezt követően megvizsgáltuk még, hogy \textbf{miként lehet morfológiai annotáló rendszerek pontosságát növelni kombinációs sémák alkalmazásával}.
Kifejlesztettünk egy olyan morfológiai egyértelműsítő rendszereket kombináló algoritmust, ami agglutináló nyelvekhez illeszkedő jellemzőket használ, így teljesítménye meghaladja más létező eljárásokét magyar nyelvű szövegen esetén.

A teljes egyértelműsítés feladatán túl, a címkéző rendszerek gyakorlati alkalmazása is kiemelkedő jelentőséggel bír.
Ennélfogva kutatásunkban tanulmányoztuk még, hogy \textbf{miként lehetséges beszélt nyelvi átiratokhoz olyan morfológiai annotáló eszközt létrehozni, mely a szakterületi kutatók munkáját képes segíteni}.
Bemutattunk egy olyan eljárást, mely a beszélt nyelvi lejegyzéseket nagy pontossággal képes morfológiai annotációval ellátni, illetve ismertettünk egy olyan módszert is, amely automatikusan képes megbecsülni gyermeknyelvi szövegek morfoszintaktikai komplexitását.

Írásunk harmadik részében elektronikus orvosi feljegyzések előfeldolgozásával foglalkoztunk.
Mindenekelőtt azt vizsgáltuk, hogy \textbf{miként lehetséges megfelelő szó- és mondatrabontó eljárásokat létrehozni létező algoritmusok továbbfejlesztésével}.
Bemutattunk egy olyan hibrid eszközt, mely szabályalapú komponenseken túl felügyelet nélküli gépi tanulásra építve azonosítja a klinikai rekordok szavait és mondatait. 
Ezt követően tanulmányoztuk még a morfológiai egyértelműsítés kérdését az orvosi szövegek tekintetében.
Megvizsgáltuk, hogy \textbf{milyen tényezők okozhatják egy klinikai dokumentumokat feldolgozó morfológiai annotáló alkalmazás legfőbb nehézségeit} illetve, hogy \textbf{a PurePos rendszer miként alkalmazható a doménre}.
A klinikai szövegek speciális tulajdonságait kihasználva, számos olyan domén adaptációs eljárást készítettünk, amelyek jelentős mértékben képesek javítani a felhasznált alaprendszer hibáin.



