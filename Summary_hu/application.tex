% !TeX spellcheck = hu_HU
A bemutatott módszerek nyelvtechnológiai alapfeladatokra adnak megoldást, így ezek komplex feldolgozóláncok alapját képezhetik.
A morfológiai címkéző algoritmusok (ld. \ref{thes:morf} téziscsoport) széles körben használhatóak információkinyerési és szövegbányászati alkalmazásokban úgy mint névelemek azonosítása, kulcsszavak kinyerése vagy dokumentumok osztályozása.
%A közreadott rendszerek elsősorban morfológiailag összetett nyelvek előfeldolgozását célozzák, így ezen nyelvek esetén képesek a feldolgo
Ezeken túl, az egyértelműsítő eljárások alábbi gyakorlati alkalmazásokról van tudomásom:
\begin{enumerate} %TODO: mik vannak még? András?
\item Laki és tsai. átrendezés-alapú angol-magyar gépifordító-rendszert \cite{Laki2013} építettek a PurePos rendszer használatával,
\item Novák és tsai. ó- és középmagyar szövegek morfológiai annotációjának elkészítéséhez használja \cite{Novak2013} a közreadott címkéző eszközt,
\item Endrédy és tsai. \cite{Endredy2014} magyar nyelvű főnévi csoport azonosítót készített, mely tartalmazza az ismertetett egyértelműsítő algoritmusokat,
\item Indig és Prószéky \cite{Indig2013} egy kötegelt helyesírás-ellenőrző programban alkalmazza az eljárást, míg
\item Prószéky és tsai. egy pszicholingvisztikai indíttatású elemzőben hasznosítják \cite{Proszeky2014} a PurePos rendszer egyes komponenseit.
\end{enumerate}

A \ref{thes:mlu}. téziscsoport magyar nyelvű beszédátiratok feldolgozásával foglalkozik. 
Az itt bemutatott módszer egy olyan speciális alkalmazása a PurePos rendszernek, mellyel ennek a doménnek a további vizsgálatát teszi lehetővé.
A \ref{thes:mlu-estimation}. tézis morfémabecslő eljárása jól használható gyermeknyelvi szövegek esetén morfoszintaktikai komplexitás automatikus mérésére, kiváltva így az időigényes manuális kalkulációt.
Továbbá Mátyus \cite{Matyus2014b} egy a gyermekek nyelvi fejlődését vizsgáló kutatásban alkalmazza a közreadott eljárásokat. 

Az utolsó téziscsoportban ismertetett eljárások zajos klinikai szövegek hatékony előfeldolgozását teszik lehetővé.
A \ref{thes:clin-segment}. tézisben bemutatott algoritmusok hatékonyan képesek szavakra és mondatokra bontani magyar nyelvű orvosi szövegeket, ezáltal lehetővé téve az azokban kódolt információ kinyerését. 
Ezen kívül a \ref{thes:clin-pos}. algoritmusai elfogadható minőségű morfológiai annotációt készítenek a klinikai rekordokhoz, így azok mélyebb elemzését teszik lehetővé.
A fenti eredményeimet egy folyamatban lévő projekt \cite{Siklosi2014,Siklosi2014mszny} hasznosítja, mely klinikai dokumentumokban rejlő rejtett összefüggések feltárását célozza  meg.


