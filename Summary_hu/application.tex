A bemutatott eljárások nyelvtechnológiai alapfeladatokra adnak megoldást, így ezek komplex feldolgozóláncok alapját képezhetik.
A morfológiai egyértelműsítő algoritmusok (ld. \ref{thes:morf} téziscsoport) széles körben használható információkinyerési és szövegbányászati alkalmazásokban úgy mint pl. névelemek azonosítása, kulcsszavak kinyerése vagy dokumentumok osztályozása.
Ezeken túl, az alábbi gyakorlati alkalmazásokról van tudomásom:
\begin{enumerate} %TODO: mik vannak még? András?
\item Laki és tsai. átrendezés-alapú angol-magyar gépifordító-rendszert \cite{Laki2013} építettek a PurePos rendszer használatával,
\item Endrédy és tsai. \cite{Endredy2014} magyar nyelvű főnévi csoport azonosítót készített, mely tartalmazza az ismertetett algoritmusokat,
\item Novák és tsai. Ó- és Középmagyar szövegek morfológiai annotációjának elkészítéséhez használja \cite{Novak2013} az közreadott eszközt,
\item Prószéky és tsai. egy pszicholingvisztikai indíttatású elemzőben hasznosítják \cite{Proszeky2014} a PurePos rendszer egyes komponenseit, míg
\item Indig és Prószéky \cite{Indig2013} pedig egy kötegelt helyesírás-ellenőrző programban alkalmazza az egyértelműsítő eljárásokat.
\end{enumerate}

A \ref{thes:mlu} téziscsoport magyar beszédátiratok feldolgozásával foglalkozik. 
Az itt bemutatott módszer egy olyan speciális alkalmazása a PurePos rendszernek, mellyel ennek a doménnek a további vizsgálatát tettem lehetővé.
Ezen túl a \ref{thes:mlu-estimation} tézis morfémabecslő eljárása jól használható pl. morfoszintaktikai komplexitás automatikus mérésére kiváltva így az időigényes manuális kalkulációt.

Az utolsó téziscsoportban ismertetett eredményeim zajos szövegek hatékony előfeldolgozását teszik lehetővé.
A \ref{thes:clin-segment} tézisben bemutatott algoritmusok hatékonyan képesek szavakra és mondatokra bontani magyar nyelvű orvosi szövegeket, ezáltal lehetővé téve az azokban kódolt információ kinyerését. 
Ezen kívül a \ref{thes:clin-pos} tézisben vázolt algoritmus megfelelő morfológiai annotációt készít a klinikai rekordokhoz, így azok mélyebb elemzését segíti.
A fenti eredményeimet egy folyamatban lévő projekt hasznosítja \cite{Siklosi2014,Siklosi2014mszny}, mely klinikai dokumentumokban rejlő rejtett összefüggések feltárását célozza  meg.


