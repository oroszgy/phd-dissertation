%TODO: valamilyen jobb első bekezdés kellene + text processing / NLP / NLT

Natural language technology eases everyday life by helping the information flow between humans and computers.
Diverse applications of the field exist which can aid e.g. writing texts, understanding foreign languages or finding relevant information pieces.
Text processing is a branch of language technology, which includes the automated analysis of textual data. %One of its major parts, text processing, deals with textual data stored on computers by annotating and representing them using linguistic concepts.
Several processing layers can be distinguished such as text segmentation, morphological parsing, syntactic parsing or semantic analysis.
Practical applications often build tools pipelining such components one after another. 
While many layers are not available for numerous languages, two preprocessing steps are indispensable for most of the cases.
Words and sentences are the basic units of text mining applications, therefore segmentation must be performed first.
Beside this, the lemmata and the \gls{pos} of words are also necessary components of such systems, thus morphological parsing is carried out next.
%Further on, pipelined architectures may easily result in erroneous output, since error propagation is often a notable phenomenon. 
%Therefore, accurate solutions are indispensable ingredients of proper applications. 

%Text segmentation is usually composed of two parts: tokenization and sentence boundary identification. 
%The first task aims to split punctuation marks from words that is usually carried out utilizing pattern matching methods.
%Next, sentences boundaries are often recognized with rules relying on abbreviations or using machine learning algorithms.
%In most of the cases, solutions for these tasks are language-specific and fine-tuned, thus resulting in accurate tools.
In most of the cases, text segmentation systems are accurate and remain robust amongst domains, hence the task is considered to be solved.
However, there are numerous scenarios (such as the case of noisy texts) on which existing approaches fail, thus posing new challenges to researchers.

Further on, \acrshort{pos} tagging is another well-researched field of \gls{nlp}, as diverse methods exist solving the problem in many languages. 
In practice, these algorithms mostly build on data-driven techniques thus restricting their applicability by the corpus they model.
%Therefore, transferring their knowledge to a new domain is often complicated without a satisfactory amount of annotated data.
Furthermore, most of the tagging methods target English first, hence ignoring serious problems caused by rich morphological systems.
In that way, disambiguating between part-of-speech labels becomes insufficient, thus full morphological tagging algorithms are required assigning full morpho-syntactic tags and computing lemmata as well.
Therefore, language technology needs preprocessing methods which can handle morphologically rich languages efficiently 
and perform well on less-resourced scenarios at the same time.

%TODO: a félkövérekkel mi legyen?
The aim of this study is twofold. 
Firstly, morphological tagging algorithms are investigated which can handle agglutinative languages and domain adaptation scenarios effectively. 
Secondly, methods suitable for less-resourced noisy domains are examined.

First, we were interested in \textbf{how existing methods can be applied for full morphological tagging of agglutinative languages yet remaining suitable for domain adaptation tasks}. 
As a result we present an accurate lemmatization method and an efficient morphological tagging chain.
%Chapter \ref{chap:tagging} focuses on many aspects of this question. 
%Section \ref{sec:tagging} investigates \textbf{how a lemmatization method can be composed relying on a morphological analyzer}. 
%This part presents a lemmatization algorithm integrating a \acrshort{ma} and employing several stochastic models to produce proper root of words.
%Moving on, this section has major focus on the full disambiguation problem, in particular on the question of \textbf{how one can create a morphological tagging architecture being accurate and also flexible enough to be used in rule-based domain adaptation tasks}.
%As a result, we introduce an accurate tool (PurePos) for full morphological tagging being customizable for diverse domains.
%The presented system is evaluated through several experiments. 
%On one hand, it is tested on a general Hungarian corpus showing its state-of-the-art accuracy.
%On the other hand, hybrid components of the tool is examined through an annotation task showing their conduciveness.
Following this, we examined \textbf{how one can improve combination schemes of full morphological taggers in order to raise the overall annotation quality}.
In that way, an architecture is introduced which fits well for agglutinative languages and improves the baselines used.

Beside tagging methods, their applications also played a central role in this study. %TODO:AR?
We were interested in \textbf{creating a proper tagger tool for speech transcripts which can help linguists in their research}.
In doing so, 
%Chapter \ref{chap:mlu} presents adaptation methods resulting in the first morphological tagging chain for spoken Hungarian.
%Following this, 
a methodology is described that can estimate morphosyntactic complexity of speech transcripts automatically.

The third part of this study deals with the preprocessing of electronic health records.
In particular, first we were interested in \textbf{how one can develop a proper text segmentation algorithm using existing methods}. 
A hybrid framework was presented using diverse symbolic and machine learning components, thus resulting in a precise tokenization and SBD method.
%Our contribution in this field is twofold. 
%First, it is shown that all available tools fail on segmenting clinical Hungarian texts.
%Next, an accurate methodology is proposed identifying sentence and token borders precisely.
Following this, we looked into the questions \textbf{what the main pitfalls of morphological taggers are which target noisy clinical texts} and \textbf{how PurePos can be adapted for tagging medical texts properly}.
%This part introduces a detailed error analysis of the tool showing that abbreviations and \gls{oov} words cause the most of the errors.
This part presents several adaptation techniques relying on domain-specific knowledge, thus improving the annotation quality significantly.

