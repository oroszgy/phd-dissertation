
The methods presented here solve basic preprocessing tasks such as text segmentation and morphological tagging. 
Since these are essential components of any language processing chain, our results can be applied in numerous fields of natural language technology. 
In general, text mining solutions and information extraction tools utilize such algorithms.
%Specific applications can employ them  in e.g. named entities recognition, keyword extraction or event identification scenarios. 
Since our methods process morphologically rich and less-resourced languages, they can be used to boost tasks involving such languages.

Concerning general tagging methods of Theses \ref{thes:morf-lemma} and \ref{thes:morf-tagging}, they have been successfully applied in several Hungarian projects.
Their applications involve the following studies:
\begin{enumerate}
\item Laki et al. \cite{Laki2013} have developed an English to Hungarian morpheme-based statistical machine translation method using PurePos,
\item Novák et al. \cite{Novak2013} have annotated Old and Middle Hungarian texts employing our methods,
\item Enrédy et al. \cite{Endredy2014} have proposed a noun phrase detection toolkit utilizing the morphological tagging tool presented,
\item Indig and Prószéky have applied \cite{Indig2013} the proposed tagger tool for a batch spelling-correction tool and
\item Prószéky et al. \cite{Proszeky2014} have built their psycho-linguistically motivated parser on top of PurePos.
\end{enumerate}

Next, Thesis group \ref{thes:mlu} presents methods and resources for analyzing transcripts of spoken language which can serve \acrshort{nlp} applications of the domain.
Besides, methods of Thesis \ref{thes:mlu-estimation} estimate morphosyntactic-complexity of children language, thus can replace the labor-intense manual work.
Furthermore, Mátyus utilizes \cite{Matyus2014b} these algorithms in her research investigating the language development of Hungarian kindergarten children.

Finally, the last (\ref{thes:clin}) Thesis group details methods for processing noisy texts effectively.
Algorithms of Thesis \ref{thes:clin-segment} segment clinical texts accurately, providing proper output for information extraction applications. 
Furthermore, lessons learned from our tagging methods could help to develop accurate text mining tools.
Beside possible applications, an ongoing project \cite{Siklosi2014,Siklosi2014mszny,Orosz2014x} processing Hungarian electronic health records benefits from the proposed methods.


