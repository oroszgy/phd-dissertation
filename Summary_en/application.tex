
Methods presented solve basic preprocessing tasks such as text segmentation and morphological tagging. 
Since these are essential components of any language processing chain, our results can be applied in numerous fields of natural language technology. 
Generally, text mining solutions and information extraction methods utilize such methods.
Concrete applications can be e.g. named entities recognition, keyword extraction or event identification. 
Since our results aim morphologically rich and less-resourced languages, they can boost task involving such languages.

Concerning general tagging methods of Theses \ref{thes:morf-lemma} and \ref{thes:morf-tagging}, they have been successfully applied in several Hungarian tools.
Applications are the following:
\begin{enumerate}
\item Laki et al. \cite{Laki2013} developed an English to Hungarian morpheme-based statistical machine translation application using PurePos,
\item Novak et al. \cite{Novak2013} annotated Old and Middle Hungarian texts with the same tool,
\item Enrédy et al. \cite{Endredy2014} proposed a noun phrase detection toolkit utilizing our morphological tagging methods,
\item Indig and Prószéky applies \cite{Indig2013} the proposed tagger tool for a batch spelling-correction tool and
\item Prószéky et al. \cite{Proszeky2014} builds their psycho-linguistically motivated parsed on top of algorithms of PurePos.
\end{enumerate}

Thesis group \ref{thes:mlu} presents methods and resources for analyzing spoken language.
Therefore, these can mainly serve NLP applications of the domain.
Besides, methods of Thesis \ref{thes:mlu-estimation} can measure morphosyntactic-complexity of children language, thus can replace the labor-intense manual work.
Furthermore, Mátyus utilizes \cite{Matyus2014b} these algorithms on her research investigating the speech of Hungarian kindergarten children.

Finally, the last (\ref{thes:clin}) Thesis group present methods for processing noisy texts effectively.
Algorithms of Thesis \ref{thes:clin-segment} segments clinical texts accurately, providing proper output for information extraction applications. 
Furthermore, lessons learned from our tagging methods could help to develop more accurate annotation tools.
Beside possible applications, an ongoing project \cite{Siklosi2014,Siklosi2014mszny} on processing Hungarian electronic health records benefit from the proposed methods.


