In the course of our work, diverse corpora have been used. 
First, general tagging methods are developed employing the Szeged Corpus \cite{Csendes2004}.
Further on, such algorithms have been tested on Old and Middle Hungarian \cite{Novak2013} texts as well.
Next, methods for speech transcripts are analyzed on the \acrshort{hukilc} corpus\cite{Matyus2014}.
Besides, two corpora have been created manually from electronic health records.
These enabled us to develop and evaluate algorithms for the clinical domain.
Concerning their usage, texts were split into training, development and test sets depending on the task nature.

As regards methods used, most of our work result in hybrid solutions.
On one hand, we build on symbolic morphological analyzers and rule-based (sometimes pattern matching) components. 
On the other hand, stochastic and machine learning algorithms are heavily utilized as well.

Morphological analyzers play central role in this work, since their usage is inevitable for morphologically complex languages.
In most of the cases we employed (adapted versions \cite{Novak2013,Orosz2013} of) Humor \cite{Novak2003,Proszeky1994} but the \acrshort{ma} of \texttt{magyarlanc} \cite{zsibrata2013magyarlanc} has been used as well.

As regards machine learning, tagging experiments are based on hidden Markov models \cite{Rabiner1989,Samuelsson1993}. 
Our approach builds on two well known tools which are Brant's TnT \cite{Brants2000} and HunPos \cite{Halacsy2007} from Halácsy et al. 
Besides, other well known methods such as $n$-gram modeling, suffix-tries and interpolation techniques are utilized as well.
Further on, the proposed combination method utilizes instance based learning \cite{Aha1991} implemented in the Weka toolkit \cite{Hall2009}.

Next, text segmentation models employ unsupervised techniques as well.
The collocation extractions measure of Dunning \cite{dunning1993accurate} is used for detecting sentence boundaries.
In fact, we base on the study of Kiss and Strunk \cite{kiss2006unsupervised} which employs scaling factors for the $\log\lambda$ ratio.

Finally, evaluation of algorithms are carried out calculating standard metrics.
Performance of taggers are measured with precision as counting correct annotation of tokens and sentences.
However, if the corpus investigated contained a considerable amount of punctuation marks they were not involved in the computation.
Improvement of taggers are usually examined calculating relative error reduction rate. 
Further on, numeric values are compared computing mean relative error \cite{Witten2011} ratio and Pearson's correlation coefficient \cite{Witten2011}.
Finally, simple classification scenarios are evaluated with standard metrics.
In doing so, precision, recall and F-score is computed for each class, while overall accuracy values are provided as well.