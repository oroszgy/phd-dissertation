\section{Preprocessing in natural language technology}

% mi a nyelvtech
% 2*2-es felosztás
% mik a legfontosabb feladatok
% általában hogyan épül fel egy feldolgozási lánc
% error propagaton

% mit nevezek preprocessingnek
% mi a text segmentation

%TODO: MM/Wikipedia koppintás... :( -- lehetne helyette a FSNLP
Natural language technology is a field of computer science and linguistics interested in interactions between computers and human (natural) languages. 
As such, it involves understanding of input deriving from humans and natural language generation. 
Beside this categorization, another one can be created considering the target of tasks: applications can handle either text or speech.

As structural layers are distinguished in natural languages, different levels of \gls{nlp} can be identified.
Concerning text processing scenarios, these can be the following:
\begin{description}
\item[Text segmentation:] basic units of texts are separated, thus token (tokenization) and sentence boundaries are recognized (referred as \gls{sbd}).
\item[Morphological parsing:] structural units of words are identified (morphological analysis), then tokens are unambiguously classified by their morphosyntactic behavior (\acrshort{pos} tagging).
\item[Syntactic parsing:] sentences are broken down into building blocks regarding their form, function or syntactic relation to each other.
\item[Semantic analysis] methods deal with the \emph{meaning} of texts involving e.g. models for formal representation or the task of word sense disambiguation.
\end{description}

Practical applications often build parsing chains pipelining such components one after another. 
While many processing layers are not available for numerous languages, two preprocessing steps are indispensable for many cases.
Since words and sentences are the basic units of text mining applications, segmentation must be the first step.
Beside this, \acrshort{pos} lemmata and labels of words are also necessary features of such systems, thus morphological parsing should be carried out next.

Moving on, pipelined architectures may easily result in erroneous output, since error propagation is often a notable phenomenon. 
Obviously, the more accurate modules are employed the better analyses are yielded, in that way, high precision of such methods are important.

\section{Is preprocessing texts a solved problem?}


% hogyan szokták megoldani a tokenizálást/SBD-t => bár megoldott, de nem mindig
% hogyan szokták megoldani a PoS taggelést => megoldott, de nem mindig
	% mi a morf. egyért. 
	% lem

Text segmentation is usually composed of two parts: tokenization and sentence boundary identification. 
The first task aims to split punctuation marks from words that is usually carried out utilizing pattern matching methods.
Next, sentences boundaries are often recognized with rules relying on abbreviations or using machine learning algorithms.
In most of the cases, solutions for these tasks are language-specific and fine-tuned, thus resulting in accurate tools.
In addition, such systems usually remain robust amongst domains, hence text segmentation is considered to be solved.
However, there are numerous scenarios (such as \acrshort{sbd} in noisy texts) on which existing approaches fail, thus posing new challenges to researchers.

Next, \acrshort{pos} tagging is another well-researched field of \acrshort{nlp}, as diverse methods exist solving the problem in many languages. 
In practice, they mostly build on data-driven algorithms which require large amount of training data.
As a result, such approaches are restricted by the corpus they model.
Moreover, transferring the knowledge of these systems (aka. domain adaptation) is often complicated and this field is still are under heavy development.

Most tagging algorithms target English first, thus ignoring serious problems caused by rich morphological systems.
For instance, agglutinative languages (such as Hungarian) bear with rich inflection systems.
Words are formed using agglutination: several affixes are joined on after another to the lemma which affecting its morphosyntactic behavior.
As a result, they need larger (morphosyntactic) tagsets compared to English . 
Furthermore, lemmatization of words cannot be carried out using simple suffix-stripping methods.
In that way, disambiguating between part-of-speech labels become insufficient, thus full morphological tagging algorithms are required assigning full morphosyntactic tags and computing lemmata as well.

All in all, language technology needs preprocessing methods which
\begin{enumerate}
\item performs well on less-resourced scenarios,
\item handles morphologically rich languages efficiently and
\item serve good base for domain adaptation approaches.
\end{enumerate}

\section{Methods for morphologically rich languages and less-resourced domains}

The aim of this study is twofold. 
Firstly, morphological tagging methods are investigated which fits well for agglutinative languages and domain adaptation scenarios. 
Secondly, methods suitable for less-resourced noisy domains are examined.

In doing so, first, we were interested in \textbf{how existing methods can be applied for full morphological tagging of agglutinative languages yet remaining suitable for domain adaptation tasks}. 
Chapter \ref{chap:tagging} focuses on many aspects of this question. 
Section \ref{sec:tagging} investigates \textbf{how a lemmatization method can be composed relying on a morphological analyzer}. 
This part presents a lemmatization algorithm integrating a \acrshort{ma} and employing several stochastic models to produce proper root of words.
Moving on, this section has major focus on the full disambiguation problem, in particular on the question of \textbf{how one can create a morphological tagging architecture being accurate and also flexible enough to be used in rule-based domain adaptation tasks}.
As a result, we introduce an accurate tool (PurePos) for full morphological tagging being customizable for diverse domains.
The presented system is evaluated through several experiments. 
On one hand, it is tested on a general Hungarian corpus showing its state-of-the-art accuracy.
On the other hand, hybrid components of the tool is examined through an annotation task showing their conduciveness.
Following this, Section \ref{sec:combination} examines \textbf{how we can improve combination schemes of full morphological taggers in order to raise the overall annotation quality}.
In doing so, an architecture is introduced which fits well for agglutinative languages and improves baselines used.


Beside fundamental methods, their applications play also a central role in this study.
We were interested in \textbf{how PurePos can help linguistic research in speech}.
In doing so, Chapter \ref{chap:mlu} presents adaptation methods resulting in the first morphological tagging chain for spoken Hungarian.
Following this, an application of the chain is described estimating morphosyntactic complexity of child speech automatically.

The third part of the dissertation (Chapter \ref{chap:clin}) deals with problems of noisy electronic health records.
In particular, Section \ref{sec:clin_segm} investigates \textbf{how one can develop proper text segmentation algorithm using existing methods}. 
Our contribution in this field is twofold. 
First, it is shown that all available tools fail on segmenting clinical Hungarian texts.
Next, an accurate methodology is proposed identifying sentence and token borders precisely.

Following this, Section \ref{sec:clin_tag} looks into the questions \textbf{what are the main pitfalls of morphological taggers aiming to process noisy clinical texts} and \textbf{how PurePos can be adapted for tagging medical texts properly}.
This part introduces a detailed error analysis of the tool showing that abbreviations and \gls{oov} words cause the most of the errors.
In addition, adaptation techniques are presented relying on domain-specific knowledge, thus improving the annotation quality significantly.

Finally, Chapter \ref{chap:sum} sums up the contributions of the study. 
In that way, theses presented highlight the main results and point to the authors related publications.


\section{Methods of investigation}
In the course of our work, diverse corpora were used. 
First, the Szeged Corpus \cite{Csendes2004} was employed for developing and evaluating general tagging methods.
Further on, these algorithms were tested on Old and Middle Hungarian \cite{Novak2013} texts as well.
Next, methods for speech transcripts were analyzed on the \acrshort{hukilc} corpus \cite{Matyus2014}.

Beside existing ones, two new corpora were created manually from electronic health records.
These texts enabled us to design algorithms for the clinical domain.
Concerning their usage, texts were usually split into training, development and test sets.

As regards methods used, most of our work resulted in hybrid solutions.
On the one hand, we built on symbolic morphological analyzers and rule-based (pattern matching) components. 
On the other hand, stochastic and machine learning algorithms were heavily utilized as well.

Morphological analyzers played a central role in our study, since their usage is inevitable for morphologically complex languages.
In most of the cases we employed (adapted versions \cite{Novak2013,NovakOMK,Orosz2013}) of Humor ~\cite{Proszeky1994,Novak2003,Proszeky2005} but the \acrshort{ma} of \texttt{magyarlanc} \cite{zsibrata2013magyarlanc} was used as well.

As regards machine learning algorithms, tagging experiments were based on hidden Markov models \cite{Rabiner1989,Samuelsson1993}. 
Our approach built on two well-known tools which are Brant's TnT \cite{Brants2000} and HunPos \cite{Halacsy2007} from Halácsy et al. 
Besides, other common methods such as $n$-gram modeling, suffix-tries and general interpolation techniques were utilized as well.
Further on, the proposed combination scheme applied instance-based learning \cite{Aha1991} implemented in the Weka toolkit \cite{Hall2009}.

Beside supervised learning, unsupervised techniques were employed as well.
Identification of sentences was performed using the collocation extractions measure of Dunning \cite{dunning1993accurate}.
In fact, we based on the study of Kiss and Strunk \cite{kiss2006unsupervised}, which employs scaling factors for the $\log\lambda$ ratio.

The effectiveness of algorithms was measured calculating standard metrics.
The performance of taggers were computed with accuracy as counting correct annotations of tokens and sentences.
However, if the corpus investigated contained a considerable amount of punctuation marks, they were not involved in the computation.
For significance tests, we used the paired Wilcoxon signed rank test as implemented in the SciPy toolkit \cite{scipy}.
Next, the improvement of taggers was examined calculating relative error rate reduction. 

Simple classification scenarios were evaluated computing precision, recall and F-score for each class.
Furthermore, overall accuracy values were provided as well.
Finally, numeric scores were compared with mean relative error \cite{Witten2011} and Pearson's correlation coefficient \cite{Witten2011}.
