%*****************************************************************************************
%*********************************** Second Chapter **************************************
%*****************************************************************************************

\chapter{Algorithms for Morphological Disambiguation}

\section{Introduction}

\todoin{
\begin{enumerate}
 \item Focused research aim: effective algorithms that can be easily domain adapted
 \item importance of the problems
 \begin{enumerate}
  \item how does lemmatization relate to the full Disambiguation problem
  \item importance of the combination 
 \end{enumerate}

 \item historical notes (?)
\end{enumerate}
}

% \subsection{Background}



% \section{Experiments}

\section{Context Dependent Statistically Driven Lemmatization}

\todoin{importance of the problem}

\subsection{Background}

\todoin{
  \begin{itemize}
    \item commonly used methods for Lemmatizing: CST, ...
    \item Uses cases for other languages
  \end{itemize}
}

\subsection{Methods}

\todoin{
\begin{itemize}
 \item suffixtree (ezerkől itt írjak vagy máshol?)
 \item TnT's theta
 \item mention n-gram methods
\end{itemize}
}

\subsection{Evaluation}

\section{Hybrid Morphological Disambiguation}

\subsection{Background}

\todoin{
  \begin{itemize}
    \item Commonly used methods for Pos Tagging: Rules, Brill, Tnt, HunPos, Maxent modells, CRF modells, SVM modells
    \item Uses cases for other languages
  \end{itemize}
}

\subsection{Methods}

\subsection{Evaluation}

\section{Combining Morphological Disambiguation Tools}

\subsection{Background}

\subsection{Methods}

\subsection{Evaluation}

\section{Summary}