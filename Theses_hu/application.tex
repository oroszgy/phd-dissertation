% A gyermekek nyelvének vizsgálata fontos szerepet tölt be a nyelvelsajátítási kutatásokban. A nyelvészek által egy előszeretettel használt metrika a nyelvi fejlődés mérésére az átlagos megnyilatkozáshossz, melyet agglutináló nyelvek esetén morfémában szokás mérni (MLU). Ennek számítása rendkívül időigényes feladat, ezért a kutatók csupán egy-kétszáz megnyilatkozás hosszát szokták megszámolni. A III.2 tézisben leírt módszer közvetlen segítséget nyújt ebben a feladatban, hiszen az algoritmus használatával, jelentősen lerövidíti ezt a munkát. Ezzel együtt a morfológiai egyértelműsítés is (III.1 tézis) elengedhetetlen az MLU számításokhoz. Bár a fent bemutatott eljárások magyar nyelvre kerültek kidolgozásra, de azok általános volta miatt más agglutináló nyelv esetén is közvetve alkalmazhatóak. 
