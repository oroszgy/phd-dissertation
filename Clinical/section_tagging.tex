Beside text segmentation, morphological tagging is also an indispensable task for information extraction scenarios. 
Even though tagging of general texts is a well-known task and considered to be solved, medical texts pose new challenges to researchers. 
In addition, English has been the main target of many studies investigating the biomedical domain up to the present time. 
Furthermore, there are just a few approaches for non- English data, neglecting agglutinative languages and particularly Hungarian.
In this work, we investigate the tagging of clinical Hungarian by adapting existing methods.% which results in an accurate tagger.

This study is structured as follows. 
Related studies are described first, then a corpus of clinical notes is presented.
%Latter texts has been collected for developing and evaluating tagging algorithms. 
In Section \ref{sec:baseline}, we introduce a baseline morphological disambiguation chain and a detailed error analysis. 
Finally, domain adaptation enhancements are presented which are being evaluated on the test corpus.

\subsection{Background}
\label{sec:biomed_tag}

Processing of biomedical texts has an extensive literature, since numerous resources are accessible for English. %TODO az introductionbe tisztázni a fogalmakat
On the contrary, much less manually annotated corpora of clinical texts are available. 
Further on, most of the work in this field has been done for English, while only a few attempts have been published for morphologically rich languages (e. g. \cite{oleynik2009performance,rost2008lessons}).

First of all, a common approach for tagging biomedical text is to train supervised sequence-classifiers. 
However,a drawback of these methods that they require manually annotated texts which are hard to create.%labor-intense human work. 
Considering the types of training material, domain-specific corpora are used either alone \cite{pakhomov2006developing,Savova2010,Smith2006} or in conjunction with a (sub)corpus of general English \cite{coden2005domain,ferraro2013improving,miller2007building}.% as training data. 
While utilizing texts only from the target domain yields acceptable performance \cite{pakhomov2006developing,Savova2010,Smith2006}, 
several experiments have shown that accuracy further increases with incorporating annotated sentences from the general domain as well \cite{barrett2011token,coden2005domain}. 
A general observation is that the more data is used from the reference domain, the higher accuracy can be achieved (e. g. \cite{pestian2004development}). 
On the contrary, Hahn and Wermter argue for training learners only on general corpora \cite{hahn2004tagging} (for German). 
Besides, there are studies on automatic selection of the training data (e. g. \cite{liu2007heuristic}).% to increase accuracy. 
What is more, there are algorithms (such as \cite{choi2012fast}) learning from several domains parallelly thus delaying the model selection decision to the decoding process. 

Next, utilization of domain-specific lexicons is another way of adapting taggers, as they can improve tagging performance significantly \cite{coden2005domain,ruch2000minimal}. 
Some studies extend existing PoS dictionaries \cite{divita2006dtagger}, while others build new specific ones \cite{Smith2006}. 
In brief, all such experiments yield significantly reduced error rates. 

Concerning tagging algorithms, researchers tend to prefer already existing applications. 
One of the most popular system is the OpenNLP toolkit\footnote{\url{http://opennlp.apache.org/}}, which is e.g. the basis of the cTakes system \cite{Savova2010}.
Further on, Brill’s method \cite{Brill1992} and TnT \cite{Brants2000} are widely used (e.g. \cite{hahn2004tagging,Savova2010,pestian2004development}) as well. 
Besides, other HMM-based solutions have also been shown to perform well \cite{barrett2011token,coden2005domain,divita2006dtagger,hahn2004tagging,pakhomov2006developing,rost2008lessons,ruch2000minimal} on biomedical texts. 

Moving on, number of experiments have revealed \cite{ferraro2013improving,ruch2000minimal,Smith2006} that domain-specific OOV words are behind the reduced performance of taggers. 
Therefore, successful methods employ either guessing algorithms \cite{barrett2011token,divita2006dtagger,rost2008lessons,ruch2000minimal,Smith2006} or broad-coverage lexicons (as detailed above). 
Beyond supervised algorithms, other approaches were also shown to be effective: Miller et al. \cite{miller2007building} use semi-supervised methods;
%\footnote{This algorithm needs raw data from the target domain, while an annotated general corpus is still used.}; 
Dwinedi and Sukhadeve build a tagger system based only on rules \cite{dwivedi8rule}; while Ruch et al. propose a hybrid system \cite{ruch2000minimal}. 
Further on, automatic domain adaptation methods (such as EasyAdapt \cite{daume2007frustratingly}, ClinAdapt \cite{ferraro2013improving} 
or reference distribution modelling  \cite{tateisi2006subdomain}) also perform well. As a drawback, they need an appropriate amount of manually annotated data from the target domain limiting their applicability. 

%First, we examine special properties of clinical notes, then a proper disambiguation methodology is being presented. 
Our method builds on a trigram tagger and a broad coverage analyzer using an extended lexicon. 
We rely on an error analysis of a baseline system (in Section \ref{sec:baseline}), while  incorporating ideas from previous studies as well.

\subsection{The clinical corpus}

First of all, special properties of clinical texts need to be considered. 
Such records are created in a special environment, thus they differ from general Hungarian in several aspects (cf. \cite{Orosz2013a,Siklosi2013b,Siklosi2012}):
\begin{inparaenum}[\itshape a\upshape)]
 \item notes contain a lot of erroneously spelled words,
 \item sentences generally lack punctuation marks and sentence initial capitalization, 
 \item measurements are frequent and have plenty of different (erroneous) forms,
 \item a lot of (non-standard) abbreviations occur in such texts and
 \item numerous medical terms are used originating from Latin.
\end{inparaenum}

As there is no corpus of clinical records available being manually annotated with morphological analyses, a new one have been created. 
These texts contain about 600 sentences extracted from notes of 24 different clinics. 
First, textual parts of the records were identified (as described in \cite{Siklosi2012}), then the paragraphs to be processed were selected randomly. 
After these, manual sentence boundary segmentation, tokenization and normalization was performed aided by methods detailed in Section \ref{sec:clinical_segmentation}. 
Manual spelling correction was carried out using system of Siklósi et al. \cite{Siklosi2013}. 
Finally, morphological disambiguation was performed: the initial annotation was provided by PurePos, then its output was checked and corrected manually. 

As regards morphological annotation of texts, clinical notes bear special properties differing from general Hungarian, which have been considered during its analysis. 
%Beside characteristics described above, the corpus has further specialties. 
These texts contain numerous \textit{x} tokens denoting multiplication, thus they are labeled as numerals. 
Latin words and abbreviations dominates sentences, which we decided to analyze regarding their meaning. 
For instance, \textit{o.} denotes \textit{szem} `eye’ thus it is tagged as a noun (\textsc{n.nom}). 
Further on, medicines names are frequent found to be almost always singular nouns. 
Finally, numerous sentences lack final punctuation marks that are not recovered in the test corpus, therefore these are not tagged either. 

\begin{table}[h]
\centering
\caption{Size of the clinical corpus created}
\label{tab:clin_corpus}
\begin{tabular}{ l @{\hspace{0.3cm}} r @{\hspace{0.3cm}} r } 
\hline
& Sentences & Tokens \\
\hline
Development set & 240 & 2230 \\
Test set & 333 & 3155 \\
\hline
\end{tabular}
\end{table}

The manually annotated corpus is split into two sets (cf. Table \ref{tab:clin_corpus}) for our experiments. 
The first part is employed for development purposes, while new methods are evaluated on the second part.
Precision is calculated considering correct full analyses of tokens, not counting punctuation marks.

\subsection{The baseline settings}
\label{sec:baseline}

We start our work by employing a baseline tagging chain, which employs components available for Hungarian. 
Humor is used to produce \emph{(morphosyntactic tag, lemma)} pairs as analyses which are then disambiguated by PurePos. 
(The output of the MA is extended with the new analysis of the \textit{x} token to fit the corpus to be tagged.)

This baseline text processing chain produced 86.61\% token accuracy on the development set, which is remarkably lower than tagging results for general Hungarian using the same components (96--98\% cf. \cite{Orosz2013b,zsibrata2013magyarlanc}). 
Measuring precision on sentences revealed that less than the third (28.33\%) of the sentences were tagged correctly. 
These fact indicates that the models used by the baseline algorithm are weak for this task. 
Therefore, errors made by the baseline algorithm are investigated to reveal common pitfalls. 

Table \ref{tab:error_types} shows that the top error class is composed of mistagged abbreviations and acronyms. 
A reason for this is that most of the abbreviated tokens are unknown to the tagger. 
Moreover, such words usually refer to medical terms originating from Latin.

Another sorts of mistakes are caused by out-of-vocabulary (OOV) words. 
These are specific to the clinical domain and often originate from Latin.
Although, this observation is in accordance with the PoS tagging results for medical English, listing of such terms' analyses is not a satisfactory solution to the problem. 
This is because the number of their inflected forms are much higher compared to English. 

Finally, domain-specific usage of words leads the tagger astray as well. 
Examples are participles being mislabeled as verbs. 
E.g.\textit{javasolt} `suggested’  and  \textit{felírt} `written’ are common words having different PoS tag distributions in this domain. 
Further on, some erroneous tags are due to the lexical ambiguity being present in Hungarian (such as \textit{szembe} which can refer to `into an eye’ or `toward/against’). 

\begin{table}[h]
\centering
\caption{Distribution of errors caused by the baseline algorithm -- dev. set}
\label{tab:error_types}
\begin{tabular}{ l r } 
\hline
Class & Frequency  \\
\hline
Abbreviations and acronyms & 49.17\% \\
Out-of-vocabulary words & 27.27\% \\
Domain-specific PoS of word forms & 14.88\% \\
% Numbers & 0.02\% \\
Other & 0.06\% \\
\hline
\end{tabular}
\end{table}

Our investigation shows that most of the errors of the baseline system can be classified into the three categories (see in Table \ref{tab:error_types}). 
We utilize this observation to enhance the baseline method by eliminating its most typical errors.

\subsection{Domain adaptation experiments}

Systematic changes are carried out to improve the accuracy of the chain. 
In doing so, each enhancement is evaluated against the development corpus during the to monitor their usefulness.

\subsubsection{Utilizing an extended morphological lexicon}
\label{sec:ma-extension}

Supervised tagging algorithms commonly use augmented lexicons reducing the number of out-of-vocabulary words (see Section \ref{sec:biomed_tag}). 
In the case of Hungarian, this must be performed at the level of the MA, since inflection is a notable phenomena. 
Extension of the lexicon has been carried out by Attila Novák \cite{Orosz2014} adding 40000 different lemmata to the analyzer. 
For this he used a spelling dictionary of medical terms \cite{Fabian1992} and a freely available list of medicines \cite{Foigazgatosag2012}.
By employing the enhanced lexicon, the OOV word ratio is reduced to 26.19\% (from 34.57\%) improving accuracy to 92.41\% (on the development set). 
Further on, the medical dictionary \cite{Fabian1992} contained numerous abbreviated tokens as well, thus the usage of the augmented analyzer also helps to decrease the number of mistagged abbreviations.

\subsubsection{Dealing with acronyms and abbreviations}

Despite the changes in Section \ref{sec:ma-extension}, numerous errors made by the improved tagger are still connected to abbreviations. 
Thus, we first examine erroneous tags of abbreviated terms, then present methods aiming to improve the performance of the disambiguation chain. %TODO: hogy írjuk, hogy HuMor vs. Humor?

A detailed examination revealed that some of the erroneous tags are due to the over-generating nature of Humor. 
To fix such problems we apply a simple filtering method. 
An analysis of a word with an attached full stop is considered to be false if the lemma candidate is not an abbreviation. 
In doing so, the overall accuracy is increased notably, reducing the number of errors on the development set by 9.20\%.
%(cf. ``Filtering'' in Table \ref{tab:abbrev_fixes}).

Another typical error type is the mistagging of unknown acronyms. 
Since PurePos do not employ features  dealing with such cases, these tokens are usually left to the suffix guesser resulting in incorrect annotation. 
In contrast, our investigation shows that acronyms should be tagged as singular nouns in most of the cases. 
Therefore, a pattern matching component is developed to fix their annotation relying on surface features.
% (see ``Acronyms'' in Table \ref{tab:abbrev_fixes}). 

Finally, the rest of the errors are mainly connected to those abbreviations which are both unknown to the analyzer and have not been seen previously. 
For this, the distribution of abbreviations' labels is compared to that of the Szeged Corpus (see Table \ref{tab:pos_distribution} below).
While there are several common properties between the two datasets (such as the ratio of adverbs), discrepancies are more significant. 
The most important difference is the proportions of adjectives: it is notably higher in the medical domain than in general Hungarian. 
Moreover, these values are more expressive if we consider that 10.85\% of the tokens are abbreviated in the development set, while the same ratio is only 0.37\% in the Szeged Corpus. 

\begin{table}[h]
\centering
\caption{Morphosyntactic tag frequencies of abbreviations -- dev. set}
\label{tab:pos_distribution}
\begin{tabular}{ l r r} 
\hline
Tag & Clinical texts & Szeged Corpus  \\ 
\hline
\scshape{n.nom} & 67.37\% & 78.18\% \\
\scshape{a.nom} & 19.07\% & 3.96\% \\
\scshape{conj} & 1.27\% & 0.50\% \\
\scshape{adv} & 10.17\% & 11.86\% \\
Other & 2.12\% & 5.50\% \\
\hline
\end{tabular}
\end{table}

Since noun tags are the most frequents amongst abbreviations, a plausible method (``UnkN'') is to assign \textsc{n.nom} to unknown ones. 
In doing so, we keep the original word forms as lemmata. 
Although this approach is really simple, it results in a surprisingly high (31.54\%) error rate reduction. 

Next, we try to approximate the abbreviations' analyses with the distribution of tags observed in Table \ref{tab:pos_distribution}. 
First, we employ (``UnkUni'') a uniform distribution of their labels.% using the development set.
In that way, \textsc{a.nom}, \textsc{a.pro}, \textsc{adv}, \textsc{conj}, \textsc{n.nom}, \textsc{v.3sg} and \textsc{v.pst\_ptcl} labels are used with equal probability as a sort of guessing algorithm.

Beside these, another reasonable method is to use maximum likelihood estimation for calculating a priori probabilities of labels (``UnkMLE''). 
In this case, relative frequency estimates are calculated for all the tags enlisted above.
Investigating the performance of enhancements aiming abbreviations (cf. Table \ref{tab:abbrev_fixes}), we found that the latter approach can increase the overall performance, but ``UnkN'' models best the task. An explanation for this surprising fact is that annotated data available can be insufficient for computing the proper distribution of labels.

\begin{table}[h]
\centering
\caption{Comparison of the techniques aiming to handle acronyms and abbreviations --  dev. set}
\label{tab:abbrev_fixes}
\begin{tabular}{ l l r } 
\hline
ID & Method &  Precision \\
\hline
0 & Medical lexicon & 90.11\% \\
1 & 0 + Filtering & 91.02\% \\
2 & 1 + Acronyms & 91.41\% \\
3 & 2 + UnkN & \underline{94.12\%} \\
4 & 2 + UnkUni & 92.82\% \\
5 & 2 + UnkMLE & 94.01\% \\
\hline
\end{tabular}
\end{table}

\subsubsection{Choosing the proper training data}

Since many studies showed (cf. Section \ref{sec:biomed_tag}) that the training data used significantly affects the result of the annotation chain, we investigated sub-corpora of the Szeged Corpus. 
Their attributes (cf. Table \ref{tab:subcorpora_attrib}) are  compared\footnote{Measurements regarding the development set were calculated manually where it was necessary.} to find a decent domain to learn from for tagging clinical Hungarian. 

\begin{table}[h]
\centering
\caption{Properties of training corpora}
\label{tab:subcorpora_attrib}
\begin{tabular}{ l r r r r r } 
\hline
\multicolumn{1}{l}{\multirow{2}{*}{Corpus}} & Avg. sent. & Abbrev.  &  Unknown & Perplexity \\
 & length & ratio &  ratio & Words & Tags \\
\hline
Szeged Corpus & 16.82 & 0.37\%\ \ \  & \underline{1.78\%} & \ \ 2318.02 & 22.56\\
\hspace{0.2cm} Fiction & 12.30 & 0.10\% & 2.44\% & 995.57 & 32.57\\
\hspace{0.2cm} Compositions & 13.22 & 0.14\% & 2.29\% & 1335.90 & 30.78\\
\hspace{0.2cm} Computer & 20.75 & 0.14\% & 2.34\% & 854.11 & 22.89\\
\hspace{0.2cm} Newspaper & 21.05 & 0.20\% & 2.10\% & 1284.89 & \underline{22.08}\\
\hspace{0.2cm} Law & 23.64 & 1.43\% & 2.74\% & \underline{824.42} & 29.79\\
\hspace{0.2cm} Short business news & 23.28 & 0.91\% & 2.50\% & 859.33 & 27.88\\
% \hline
Development set & 9.29 & 10.85\% & -- & -- & -- \\
\hline
\end{tabular}
\end{table}

First of all, an important property is the lengths of sentences. 
Texts having shorter sentences tend to have simpler grammatical structure, while longer sentences are grammatically more complex. 
Further on, clinical texts have a vast amount of abbreviations, thus their ratio can also serve as a relevant metric. 

In addition, the accuracy of a tagging system depends heavily on the ratio of unknown words, therefore their proportions are calculated. 
For this, we measured OOV ratio of the development set compared to vocabularies of training corpora (see Table \ref{tab:subcorpora_attrib}). 

Besides, we also compute perplexity, since it can measure the similarity of different texts \cite{kilgarriff1998measures}. 
Calculation is carried out as follows: trigram models of word and tag sequences are trained on each corpus using Kneser-Ney smoothing, then all of them are evaluated on the development set\footnote{We used the SRILM toolkit \cite{stolcke2002srilm} for training models and measuring perplexity.}.

Similarity scores show that neither parts of the Szeged Corpus contain as much abbreviated terms as clinical texts have. 
Likewise, sentences written by clinicians are significantly shorter than that of the Szeged Corpus. 
Neither the calculations above, nor the ratio of unknown words suggest using any subcorpora for training. 
However, the perplexity scores contradict: sentences from the law domain share the most phrases with clinical notes, while news texts have the most similar grammatical structures. 

\begin{table}[h]
\centering
\caption{Evaluation of the tagger using the subcorpora as training data -- dev. set}
\label{tab:eval_subcorpora}
\begin{tabular}{ l r } 
\hline
Corpus & Morph. disambiguation accuracy \\
\hline
Szeged Corpus & \underline{94.73\%} \\
\hspace{0.2cm} Fiction & 92.01\% \\
\hspace{0.2cm} Compositions & 91.97\% \\
\hspace{0.2cm} Computer & 92.73\% \\
\hspace{0.2cm} Newspaper & \underline{93.29\%} \\
\hspace{0.2cm} Law & 92.17\% \\
\hspace{0.2cm} Short business news & 92.69\% \\
\hline
\end{tabular}
\end{table}

In doing so, we employed the previously enhanced tagger (cf. Section \ref{}). 
Since similarity measurements are not in accordance with each other, all sub-corpora are verified as a training data for tagging clinical texts. 
Our results (in Table  \ref{tab:eval_subcorpora}) on the development set indicate that training on a  subcorpus does not improve the performance. % , however the whole data yields a better tagger.
This observation led us to use utilize the whole Szeged corpus to train the morphological tagger.

\subsection{Evaluation}

The improved tagging chain (cf. Table \ref{tab:improvements}) is evaluated by investigating part-of-speech tagging, lemmatization and the whole morphological tagging performance.% of each enhancement presented.

\begin{table}[h]
\centering
\caption{Evaluation of the improved tagger -- test set}
\label{tab:improvements}
\begin{tabular}{ l l r r r} 
\hline
ID & Method & PoS tagging & Lemmatization & Morph. disambig. \\
\hline
0 & Baseline system & 90.57\% & 93.54\% & 88.09\% \\
1 & 0 + Lexicon extension & 93.89\% & 96.24\% & 92.41\% \\
2 & 1 + Handling abbreviations & \underline{94.81\%} & \underline{97.60\%} & \underline{93.73\%} \\
3 & 2 + Training data selection & 94.25\% & 97.36\% & 93.29\% \\
\hline
\end{tabular}
\end{table}

First of all, results show that the baseline method annotated almost 12\% of the tokens erroneously while our enhancements raised the ceiling of the full morphological tagging accuracy to 93.73\% by eliminating almost half (47.36\%) of the errors. 
Next, precision scores indicate that the error rate reduction is mainly due to the extended lexicon and the proper handling of abbreviations. 
However, the small amount of the annotated data inhibited better modeling of abbreviated terms.
Considering training data, we did not manage to find a better (sub)corpus for tagging clinical Hungarian than the whole Szeged Corpus.% was able to outperform the whole as a training corpus. 
In sum, our improvements yielded a system having satisfactory performance for parsing morphologically clinical texts.
