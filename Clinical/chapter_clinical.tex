
\section{Introduction}

Hospitals produce a huge amount of clinical notes that have been solely used for archiving purposes and have generally been inaccessible to researchers. 
However, application of recent \acrshort{nlp} technology can make accessible the hidden knowledge of archived records, thus boosting medical research. 
An example is the English cTAKES system \cite{Savova2010}, which can recognize important medical concepts in clinical free-text documents. 
Beside extracting diseases, symptoms and treatments, the tool is also able to identify various relations between them.
This automatically extracted structured knowledge can be e.g. used to 
\begin{itemize}
	\item create family histories from medical records,
	\item find document discrepancies,
	\item spot similar cases and
	\item build question answering or semantic search systems.
\end{itemize}
While developing text processing tools for medicians is an emerging field in many developed countries, less-resourced languages lack such resources.

To be able to extract information from medical texts, they must be preprocessed properly. 
Firstly, adequate text segmentation methods 
%\footnote{While the term \emph{text segmentation} is widely used for diverse tasks, in our work it denotes the process of dividing text into tokens and sentences.} 
are required for finding token and sentence boundaries. 
Secondly, morphological tagging is an indispensable step for information extraction scenarios. 
Considering the case of Hungarian, there are only a few studies on processing medical records. 
Recently, Siklósi et al. \cite{Siklosi2012,Siklosi2013} have presented a system that is able to correct spelling errors in clinical notes. 
Their system uses a mixture of language models to generate correction candidates, however it focuses only on correctly segmented words. 
Beside error correction, an abbreviation resolution method was also presented by them \cite{Siklosi2013b}, however, problems of text segmentation and  morpho-syntactic tagging are still untouched. 
Furthermore, as far as we know, no study investigates such preprocessing tasks on Hungarian clinical texts. 

Therefore, this chapter presents accurate preprocessing algorithms for noisy medical texts.
Methods were developed and presented for only Hungarian, but they are designed in a way to perform well on other morphologically complex languages as well. 
Firstly, an effective method is introduced for detecting sentence and token boundaries.
The presented system builds on well-known tokenization rules boosting them with the knowledge of a morphological analyzer and the output of an unsupervised filtering algorithm.
Secondly, tagging experiments are presented yielding a viable morphological tagger for Hungarian electronic health records. 
The proposed tool builds on PurePos fixing its most common errors regarding the domain.

\section{Segmenting texts of electronic health records}\label{sec:clin_segm}
Error propagation in a text processing chain is usually a notable problem, therefore accurate segmentation methods are essential to parse texts properly.
Moreover, notes written by doctors are extremely noisy containing errors which inhibit the application of existing tools.

Even though tokenization and sentence segmentation methods perform well on general Hungarian, they have serious difficulties on clinical records.
These originate in special properties of such texts involving
\begin{enumerate} %[\itshape a\upshape)]
 \item typing errors (i.e. mistyped tokens, nonexistent strings of falsely concatenated words) and
 \item nonstandard usage of the language.
\end{enumerate}
While errors of the first type can be corrected easily with a rule-based tool, others need advanced methods. 

In this section, a hybrid approach to segmentation of noisy clinical records is presented. 
The method consists of two phases: first, tokens are partially segmented; then, sentence boundaries are identified.
We start with detailing the background of our research and introducing resources used.
Then, key elements of tokenization and \gls{sbd} algorithms are described. 
Finally, our system is systematically evaluated on a gold standard corpus showing its high performance.

\subsection{Previous approaches on text segmentation}

Even though, numerous studies deal with English medical texts, only few attempts have been made (cf. \cite{Siklosi2012,Siklosi2013,Siklosi2013b}) for Hungarian. 
Further on, the task of detecting sentence and word boundaries in health records is often a neglected issue.
Studies for Hungarian pay almost no attention on segmenting texts, while
most of the approaches for English ignore this question. 
First, we review general tokenization and sentence boundary detection techniques first, then describe their application on the biomedical domain.

The task is often composed of several parts: normalization (when necessary), tokenization, and sentence boundary detection.  
Although, these are generally performed one after another, there are approaches (e.g. \cite{zhu2007unified}), where tokenization and \acrshort{sbd} are treated as a unified tagging problem (such as in \cite{mikheev2000tagging}). 
Further on, handling of abbreviations is often involved in the segmentation process, since their identification helps to detect sentence and token boundaries.

As regards tokenization, it is generally treated as a simple engineering problem\footnote{In the case of alphabetic writing systems.} cutting off punctuation marks from words. 
On the contrary, \acrshort{sbd} is a rather researched topic. 
As Read et al. summarize \cite{read2012sentence}, sentence segmentation approaches fall into three classes: 
\begin{enumerate}
 \item rule-based methods employing domain- or language-specific knowledge (such as abbreviations); 
 \item supervised machine learning approaches, which may not be robust amongst domains (being specialist on the training corpus); and
 \item unsupervised learning methods extracting their knowledge from raw unannotated data. 
\end{enumerate}

As regards \acrshort{ml} attempts, one of the first pioneers was Riley \cite{riley1989some} who employed decision-tree learners to classify full stops. 
He utilized mainly lexical features (such as word length or case) to compute the probability of a word being sentence-initial or sentence-final. 
Next, Palmer et al. presented \cite{palmer1997adaptive} the SATZ system, employing supervised learning algorithms. 
Since this tool can be easily adjusted through surface and syntactic features, it has been successfully applied to several European languages. 
Further on, the maximum entropy learning approach was used as well to the task by Reynar and Ratnaparkhi \cite{reynar1997maximum}. 
Their system classifies tokens containing `.', `?' or `!' characters utilizing contextual features and abbreviation lists. 
Recently, a similar approach has been presented by Gillick \cite{gillick2009sentence} for English, using applying vector machines and resulting in state-of-the-art performance.

Beside machine learning approaches, rule-based methods are also commonly applied for these tasks. 
E.g. Mikheev presents \cite{mikheev2002periods} a small set of rules for detecting sentence boundaries (\acrshort{sb}) with a high accuracy. 
In another system presented of him \cite{mikheev2000tagging}, the latter method is integrated into a \acrshort{pos} tagging framework enabling the classification of punctuation marks. 
In doing so, they can be labeled as sentence boundaries, abbreviations or both. 
Moving on, Kiss and Strunk have introduced \cite{kiss2006unsupervised} an unsupervised method for sentence boundary detection in 2009. 
Their tool, Punkt uses scaled log-likelihood ratio for deciding whether a \emph{(word, full stop)} pair is a collocation or not.

Although tokenization and \acrshort{sbd} tasks are well established fields of natural language processing, there are only a few attempts aiming medical texts. 
These sentence segmentation attempts fall into two classes: some develop rule-based systems (e.g. \cite{XuSDJWD10}), while most of the studies employ supervised machine learning algorithms (such as \cite{apostolova2009automatic,cho2002text,Savova2010,taira2001automatic,tomanek2007sentence}).
Latter approaches usually train \acrlong{maxent} or \acrshort{crf} learners, thus large handcrafted training corpora are essential. 

Training data used are either domain-specific or general. 
In practice, domain-specific knowledge yield better performance, however Tomanek et al.  \cite{tomanek2006reappraisal} argue on using only a general-purpose corpus. 
Their results indicate that the domain of the training corpus is not critical (at least for German).

As regards Hungarian, there are only two tools available. 
Huntoken \cite{Halacsy2004} is an open source system based on Mikheev’s system, while \texttt{magyarlanc} \cite{zsibrata2013magyarlanc} has an adapted version of MorphAdorner’s rule-based tokenizer \cite{kumar2009monk} and sentence splitter. 
Both of them employ general-purpose methods utilizing language- and domain-specific rules and dictionaries.

This study introduces new methods for segmenting Hungarian clinical texts.
For this, special properties of the target domain is investigated first by creating a manually segmented corpus. 
Then, a method is presented which combines high precision rules with unsupervised learning.

\subsection{Evaluation metrics}
\label{sec:metric}

There is no metric commonly used to measure segmentation methods, therefore we review existing ones.
On the one hand, researchers specializing in machine learning approaches prefer to calculate precision, recall and $F$-measure. 
However, these scores are often used for measuring the correctness of sentence boundaries only.
On the other hand, studies on speech recognition prefer to compute NIST and Word Error Rate. 

Recently, Read et al. have reviewed \cite{read2012sentence} the state-of-the-art of text segmentation proposing a unified metric to compare different approaches. 
Their method allows measuring sentence boundaries at any position labeling characters as sentence-finals or non sentence-finals. 
In doing so, simple accuracy measures the performance. 

Our study builds on their results \cite{read2012sentence} adapting it to the full segmentation task of Hungarian clinical texts. 
In that way, we consider the corpus as a sequence of characters and empty strings and treat text segmentation as a single classification problem. 
Therefore, all the entities (either characters or empty string between them) can be labeled with one of the following tags: 
\begin{description}
 \item[$\langle$T$\rangle$] --  if the entity is a token boundary,
 \item[$\langle$S$\rangle$] -- if it is a sentence boundary,
 \item[$\langle$None$\rangle$] -- otherwise.
\end{description}
This classification scheme enables us to compute accuracy of the unified segmentation task. 
Moreover, it allows computing further common metrics.

Since it is important to measure each subsystem's correctness, precision and recall were computed for both word tokenization and \acrlong{sbd}. 
Further on, word segmentation was evaluated with $F_1$, while $F_{0.5}$ was computed for the \acrshort{sbd} task. 
(The latter calculation makes precision more important than recall.)
We employed the latter metric, because an erroneously split sentence may cause information loss\label{sec:loss}, while statements might still be extracted from longer multi-sentence text. 

\subsection{Clinical texts used}
\label{sec:clin_corpus}

A gold standard corpus of clinical texts was collected and manually corrected in order to develop and evaluate segmentation approaches.
This process involved several steps involving normalization, as such texts are full with diverse mistakes. 
In doing so, we had to deal with the following types of errors\footnote{Text normalization steps were carried out employing regular expressions.}:
\begin{enumerate}
 \item doubly converted characters, such as `\&amp;gt;',
 \item typewriter problems (e.g. `1' and `0' is written as `l' and `o'),
 \item dates and date intervals being in various formats with or without necessary whitespaces (e.g. `2009.11.11', `06.01.08'),
 \item missing whitespaces between tokens usually introduced various types of errors, such as:
 \begin{enumerate}
  \item measurements were erroneously attached to quantities (e.g. `0.12mg'),
  \item lack of whitespace around punctuation marks (e.g. `töröközegek.Fundus:ép.'),
 \end{enumerate}
 \item various formulation of numerical expressions.
\end{enumerate}
 
To investigate possible pitfalls, the gold standard data is split into two parts of equal sizes: a development and a test set containing 1,320 and 1,310 sentences respectively. 
The first part was used to identify typical problems and to develop the segmentation methods, while the second one was employed to evaluate the results. 

As initial step, the distributions of abbreviations, punctuation marks and capitalization is investigated in these texts to reveal possible difficulties. 
Comparing our data with a corpus of general Hungarian (Szeged Corpus \cite{Csendes2004}) uncovers numerous discrepancies: 
\begin{enumerate}
 \item 2.68\% of tokens found in clinical corpus sample are abbreviations while the same ratio for general Hungarian is only 0.23\%; 
 \item sentences taken from the Szeged Corpus almost always end in a sentence final punctuation mark (98.96\%), while these are totally missing from clinical statements in 48.28\% of the cases; 
 \item sentence-initial capitalization is a general rule in Hungarian (99.58\% of the sentences are formulated properly in the Szeged Corpus), but its usage is not common in the case of clinicians (12.81\% of the sentences start with a word that is not capitalized); 
 \item the amount of numerical data is notable in medical records (13.50\% of sentences consist exclusively of measurement data and abbreviations), while text taken from the general domain rarely contains statements that are full of measurements. 
\end{enumerate}

%Concidering that general-purpose tools may have difficulties with texts from this domain. 
%Such methods usually builds on features such as word capitalization and presence of punctuation marks, however their usage is significantly differ in our case. 



\subsection{Segmentation methods}
\label{sec:clinical_segmentation}

Our system is built up from several components (cf. Figure \ref{fig:clin-segment-arch}). 
First, a symbolic method (referred as the baseline) marks word and sentence boundaries\footnote{Rules and heuristics used are formulated investigating the development corpus.} seeking for full stops. 
Then, an unsupervised filtering method extends its output.
Finally, rules employing capitalization yields further sentence boundaries.

\begin{figure}[H]
  \centering
  \includegraphics[scale=0.2]{Clinical/clin_segm_arch.png} 
  \caption{The architecture of the proposed method}
  \label{fig:clin-segment-arch}
\end{figure}

\subsubsection{Rule-based word tokenization and sentence segmentation}

Our baseline method is composed of two parts. 
First, it tokenizes words (BWT) using regular expressions implemented in standard tokenizers. %then some of the sentence boundaries are classified.
However, this algorithm does not try disambiguate all words containing periods, as it would need the proper recognition of domain-specific abbreviations as well. %Therefore it only marks just a few boundaries

Further on, sentence segmentation (BSBD) is carried out minimizing information loss (as described in section \ref{sec:loss}). 
In that way, the method tries to avoid of making false-positive errors by splitting sentences only if there is a high confidence of success. 
We found that such cases are, when:
\begin{enumerate} 
 \item a period or exclamation mark directly follows another punctuation mark\footnote{Question marks are not considered as sentence-final punctuation marks, since they generally indicate a questionable finding in clinical texts.};
 \item a line starts with a full date, and is followed by other words (The last white-space character before the date is marked as a \gls{sb}.);
 \item a line begins with the name of an examination followed by a semicolon and a sequence of measurements.
\end{enumerate}

Realization of these simple observations yield 100\% precision and 73.38\% recall on tokenization considering the development set. 
The corresponding values for detecting ends of sentences are 98.48\% and 42.60\% respectively. 
As less than half of the sentence boundaries are discovered, this method needs further improvements.
In addition, a deeper analysis unfolded that the tokenization module has difficulties only with sentence final periods. 
We found that these sorts of errors are effects of the conservative tokenization algorithm, which left several words with punctuation mark attached ambiguous.
%In that way, the algorithm is adjusted incorporating unsupervised learning.

%\subsubsection{Improvements on sentence boundary detection}\label{sec:improvements}
\subsubsection{Unsupervised sentence boundary classification}

In order to enhance the baseline method we considered investigating two kinds of indicators that are usually employed in such scenarios:
%There are two indicators generally used for detecting sentence boundaries.
\begin{description}
 \item[Periods:] when a punctuation mark ($\bullet$) is attached to a word, a sentence boundary is found for sure only if the token is not an abbreviation.
 \item[Capitalization:] if a word starts with a capital letter and it is neither part of a proper name nor of an acronym, it indicates the beginning of a sentence.
\end{description}
Considering our case:
\begin{enumerate}
\item clinicians introduce new abbreviations frequently which are not part of the standard, therefore a proper list cannot be collected easily, further on, 
\item Latin words, abbreviations and subclauses are sometimes capitalized by mistake, thus they are neither reliable information sources.
\end{enumerate}
In addition, numerous sentence boundaries lack both of these indicators (as shown in Section \ref{sec:clin_corpus}). %Therefore, these are not directly applicable in our case. 

Even though these features do not function regularly, they can still be utilized.  %without a list of possible abbreviations. 
It is enough to find \emph{evidence} for the separateness of a word and the subsequent full stop to classify a position as a sentence boundary. 
For this, we employed the idea of Kiss and Strunk \cite{kiss2006unsupervised} and adapted it for clinical texts.

Their method (log-likelihood ratio) was first applied to identify collocations \cite{dunning1993accurate}, however, they managed to adjust it for the \acrshort{sbd} problem recently (cf. Punkt\cite{kiss2006unsupervised}). 
The Punkt tool considers abbreviations as collocations of words and periods, thus evaluating them using a modified log-likelihood ratio.
In practice, this is formulated via a null hypothesis \eqref{eq:h0} and an alternative one \eqref{eq:ha}. 

\begin{equation} \label{eq:h0}
H_0: P(\bullet|w) = p = P(\bullet|\neg w)
\end{equation}

\begin{equation} \label{eq:ha}
H_A: P(\bullet|w) = p_1 \neq p_2 = P(\bullet|\neg w) 
\end{equation}

\begin{equation} \label{eq:loglambda}
\log \lambda = -2 \log \frac{L(H_0)}{L(H_A)}
\end{equation}


In these formulae $H_0$ expresses the independence of a \emph{(word, $\bullet$)} pair, while $H_1$ formulates that their co-occurrence is not just by chance. 
Their ratio is measured by the $\log \lambda$ score \eqref{eq:loglambda}, which is found to be asymptotical to $\chi^2$. %, thus  can be applied as a statistical test \cite{dunning1993accurate}. 
Kiss and Strunk recovered that this algorithm performs poorly (in terms of precision) for identifying abbreviation, thus they introduced several scaling factors \cite{kiss2006unsupervised}. 
In doing so, their method %lost the asymptotic relation to $\chi^2$ and 
became a simple filtering algorithm.

We improved their approach in numerous ways. 
First of all, the inverse score ($iscore=1/log\lambda$) was used as a base, since it helps to find candidates co-occurring only by chance. 
Moving on, we introduced further scaling factors reviewing that of Punkt and adapting them to match the characteristics of the target domain.

First of all, the first factor of the Punkt system cannot be directly applied in our case.
Counts and count ratios alone do not indicate properly alone whether a token and the period is related in a clinical record, since several sorts of abbreviations occur with relative low frequencies. 


Next, lengths of words ($len$) was also used in Punkt to indicate abbreviations well. 
They could help in our case, since shorter tokens tend to be abbreviations, while longer ones do not. 
Therefore, we reformulated the original function to penalize short words and reward longer ones. 
Having a medical abbreviation list of almost 200 \label{sec:abbrev} elements\footnote{The list is gathered with an automatic algorithm on the development corpus using word shape properties and frequencies. The most frequent elements are manually verified and corrected.} 
we found that more than 90\% of the abbreviations are shorter than three characters. 
This fact led us to formulate the scaling factor as in \eqref{eq:s_l}. 
In doing so, this enhancement can also decrease the score of a bad candidate, which distinguishes it from the original formula of Kiss and Strunk.

\begin{equation} \label{eq:s_l}
S_{length}(iscore)= iscore \cdot \exp{(len/3-1)}
\end{equation}

Recently, Humor ~\cite{Proszeky1994,Novak2003,Proszeky2005}  has been extended with the content of a medical dictionary \cite{Orosz2013}. 
What is more, the analyzer is able indicate whether analyses refers to abbreviations.
Therefore, its output is used to enhance the sentence segmentation algorithm.  

An indicator function was introduced (cf. Equation \ref{eq:sign}) to utilize its output deciding whether a word can be an abbreviation or not.
Since, the morphological lexicon used is a well-established resource, our application could rely on it with high confidence.
Therefore, the factor formulated (cf. Equation \ref{eq:s_m}) uses larger weights compared to others. 
This method raises the score of a full word, decreases that of an abbreviation, while values of unknown words are left as they were.

\begin{equation}\label{eq:sign}
 indicator_{morph}(word) =
  \begin{cases}
   1  & \text{if $word$ has an analysis of a known full word} \\
   -1 & \text{if $word$ has an analysis of a known abbreviation} \\
   0  & \text{otherwise}
  \end{cases}
\end{equation}

\begin{equation} \label{eq:s_m}
S_{morph}(iscore)= iscore \cdot \exp{( indicator_{morph} \cdot len^2)}
\end{equation}

%Besides, utilization of an additional attribute of tokens can also boost the segmentation process.
Hyphens are generally not present in abbreviations but rather occurs in full words. 
Relying on this observation, $iscore$ was adjusted \eqref{eq:s_h} with a further indicator function:
$indicator_{hyphen}$ outputs $1$ only if the word contains a hyphen. 

\begin{equation} \label{eq:s_h}
S_{hyphen}(iscore)= iscore \cdot \exp{(indicator_{hyphen} \cdot len)}
\end{equation}

\begin{equation} \label{eq:scaling}
S = S_{length} \circ S_{morph} \circ S_{hyphen}
\end{equation}

Scaled $\log \lambda$ (cf. $S(iscore)$ in Equation \ref{eq:scaling}) is calculated for all \emph{(word, $\bullet$)} pairs not followed by any other punctuation mark. 
If this value is found to be higher than a threshold, the period is regarded as a sentence boundary and it is detached.\footnote{Threshold value is empirically set to $1.5$.} 
Otherwise, the joint token is treated as an abbreviation.

To investigate the improvement of our method, it was pipelined with the BSBD module producing 77.14\% recall and 97.10\% precision on the development set. 
Accuracy values show significant improvements, however they also indicate that many sentence boundaries are still not found.

\subsubsection{Rules on capitalization}

To further improve the method, capitalization properties of words were also utilized. 
We developed a rule-based component to decide whether a capitalized words can start a sentence or not.
Good \acrshort{sb} candidates of such tokens are the ones not following a non sentence terminating\footnote{Sentence terminating punctuation marks are the period and the exclamation mark for this task.} punctuation, and are not part of a named entity. 
Therefore, sequences of capitalized words are considered to be named entities and omitted as a first step. 
Then, the rest of the candidates are processed by Humor.
We employed a simple heuristic for detecting sentence boundaries:
if a word does not have a proper noun analysis but is capitalized, it is marked as the beginning of a sentence.  
Our results on the development set show that this component also enhances the BSBD: it increases recall to 65.46\% while keeps precision high (96.37\%). 

\subsection{Evaluation}

% Our hybrid algorithm has been developed using the development set, thus it is evaluated against the rest of the data. 
% %As the starting point of our comparison, we present the accuracy values of the preprocessed input text and the baseline tokenized one. 

\begin{table}[H]
\centering
\caption{Accuracy of the input text compared with segmented ones}
\label{tab:base}
\begin{tabular}{ l  r } 
\hline
& Accuracy \\ 
\hline
Raw corpus  & 97.55\% \\
BSBD & 99.11\% \\
\hspace{0.2cm}+ LLR & 99.72\% \\
\hspace{0.2cm}+ CAP & 99.26\% \\
\hspace{0.2cm}+ LLR + CAP & 99.74\% \\
\hline
\end{tabular}
\end{table}

Evaluation is presented for each components showing their accuracies (cf. Table \ref{tab:base}).
First, our improvements are compared to both the baseline module and the raw preprocessed corpus.
The unsupervised \acrshort{sbd} algorithm is marked with LLR\footnote{Referring to the term log-likelihood ratio.}, while the last component is indicated by CAP.
Results show high accuracies for the overall segmentation task, furthermore the scores of the raw corpus is relatively high.
This indicates that the metric applied is not well balanced.

Therefore, their improvements are also investigated calculating \acrlong{err} ratios (in Table \ref{tab:reduction}). 
Comparison is carried out measuring enhancements over the baseline method (BSBD) showing that both of the components improves the segmentation method.

\begin{table}[H]
\centering
\caption{Error rate reduction over the baseline method}
\label{tab:reduction}
\begin{tabular}{ l  r } 
\hline
& Error rate reduction\\
\hline
LLR & 58.62\% \\
CAP & 9.25\% \\
LLR + CAP & 65.50\% \\
\hline
\end{tabular}
\end{table}

Considering sentence boundaries only, a more detailed analysis is got by computing precision, recall and $F_{0.5}$ values (in Table \ref{tab:prec_rec}). 
Data shows that each component significantly increases the recall, while precision is just barely decreased. 
Finally, the combined hybrid algorithm\footnote{It is the composition of the BWT, BSBD, LLR and CAP components.} brings significant improvement over the well-established baseline.

\begin{table}[H]
\centering
\caption{Evaluation of the proposed sentence segmentation algorithm compared with the baseline}
\label{tab:prec_rec}
\begin{tabular}{ l r r  r  } 
\hline
& Precision & Recall & $F_{0.5}$ \\
\hline
Baseline & 96.57\% & 50.26\% & 81.54\%  \\
+ LLR & 95.19\% & 78.19\% & 91.22\% \\
+ CAP & 94.60\% & 71.56\% & 88.88\% \\
+ LLR + CAP & 93.28\% & 86.73\% & \underline{91.89\%} \\
\hline
\end{tabular}
\end{table}


While our approach focuses on the sentence identification task, we showed that it improves word tokenization as well. 
Table \ref{tab:tok_eval} presents measurements on word segmentation indicating that our enhancements resulted in a higher recall, while they did not decrease precision notably. \label{sec:eval}

%0.9974	0.7494	0.8558
%0.9854	0.9532	0.9690
%0.9974	0.7494	0.8558
%0.9854	0.9532	0.9690

\begin{table}[H]
\centering
\caption{Comparing tokenization performance of the new tool with the baseline one}
\label{tab:tok_eval}
\begin{tabular}{ l r r r} 
\hline
& Precision & Recall & $F_{1}$ \\
\hline
Baseline & 99.74\% & 74.94\% & 85.58\%  \\
Hybrid system & 98.54\% & 95.32\% & \underline{96.90\%} \\
\hline
\end{tabular}
\end{table}

Besides, the proposed method was compared with freely available tools as well (
%There are only two applications for Hungarian text segmentation: 
\texttt{magyarlanc} and Huntoken)
The latter system can be slightly adapted to a new domain by providing a set of abbreviations, thus two versions of it were evaluated. 
The first one employs a set of general Hungarian abbreviations (HTG), while the second one utilizes an extended dictionary\footnote{As described in section \ref{sec:abbrev}.} containing medical ones as well (HTM). 
Further on, Punkt \cite{kiss2006unsupervised} and the OpenNLP \cite{Baldridge2002} toolkit\footnote{The general-purpose Szeged Corpus was used as training data for the \acrlong{maxent} learning method.} were also involved in our comparison. 
The latter tool is a general framework of \acrlong{maxent} methods, hence it could be applied to detect sentence boundaries as it is presented in \cite{reynar1997maximum}.


\begin{table}[H]
\centering
\caption{Comparison of the proposed hybrid \acrshort{sbd} method with competing ones}
\label{tab:comparison}
\begin{tabular}{ l r r r} 
\hline
& Precision & Recall & $F_{0.5}$ \\
\hline
\texttt{magyarlanc} & 72.59\% & 77.68\% & 73.55\% \\
HTG & 44.73\% & 49.23\% & 45.56\% \\
HTM & 43.19\% & 42.09\% & 42.97\% \\
Punkt & 58.78\% & 45.66\% & 55.59\%  \\
OpenNLP & 52.10\% & 96.30\% & 57.37\% \\
Hybrid system & 93.28\% & 86.73\% & \underline{91.89\%} \\
\hline
\end{tabular}
\end{table}

Results in Table \ref{tab:comparison} show that general segmentation methods fail on Hungarian clinical notes in contrast to our new algorithm. 
The hybrid approach presented bears with both high precision and recall, providing accurate sentence boundaries.
While it was found that the \acrshort{maxent} approach has decent recall as well, boundaries marked by it are false positives in almost half of the cases. 
Further on, rules of \texttt{magyarlanc} seem to be robust, but the overall low performance inhibits its application for clinical texts. 
Finally, other tools do provide not just low recalls, but their precision values are still around 50\% limiting their applicability. 

In sum, the presented segmentation method successfully deals with several sorts of imperfect sentence and word boundaries.
It performs better in terms of precision and recall than competing ones, achieving 92\% of $F_{0.5}$-score. 
Finally, our results indicates that the new hybrid algorithm is a proper tool for processing clinical Hungarian.




\pagebreak

\section{Morphological tagging of clinical notes}\label{sec:clin_tag}
Beside text segmentation, morphological tagging is also an indispensable task for information extraction scenarios. 
Even though tagging of general texts is well-known and considered to be solved, medical texts pose new challenges to researchers. 
In addition, English has been the main target of many studies investigating the biomedical domain up to the present time. 
Furthermore, there are just a few approaches for non- English data, neglecting agglutinative languages and particularly Hungarian.

This section investigates the tagging of clinical Hungarian by adapting existing methods. % which results in an accurate tagger.
Our work is structured as follows. 
Related studies are described first, then a corpus of clinical notes is presented.
%Latter texts has been collected for developing and evaluating tagging algorithms. 
%In Section \ref{sec:baseline}, we introduce a baseline morphological disambiguation chain and a detailed error analysis. 
Finally, domain adaptation enhancements are introduced which are then evaluated on the test corpus.

\subsection{Background}
\label{sec:biomed_tag}

In general, tagging of biomedical texts has an extensive literature, since numerous resources are accessible for English. 
On the contrary, much less manually annotated corpora of clinical texts are available. 
Further on, most of the work in this field was done for English, while only a few attempts were published for morphologically rich languages (e.g. \cite{oleynik2009performance,rost2008lessons}).

First of all, a common approach for tagging biomedical text is to train supervised sequence-classifiers. 
However, a drawback of these methods is that they require manually annotated texts which are hard to create. %labor-intense human work. 
Considering the types of training material, domain-specific corpora are used either alone \cite{pakhomov2006developing,Savova2010,Smith2006} or in conjunction with a (sub)corpus of general English \cite{coden2005domain,ferraro2013improving,miller2007building}. % as training data. 
While utilizing texts only from the target domain yields acceptable performance \cite{pakhomov2006developing,Savova2010,Smith2006}, 
several experiments have shown that accuracy further increases with incorporating annotated sentences from the general domain as well \cite{barrett2011token,coden2005domain}. 
It is shown (e.g. \cite{pestian2004development}) that the more data is used from the reference domain, the higher accuracy can be achieved. 
However, Hahn and Wermter argue for training learners only on general corpora \cite{hahn2004tagging} (for German). 
Besides, there are studies on automatic selection of the training data (e. g. \cite{liu2007heuristic}). % to increase accuracy. 
What is more, there are algorithms (such as \cite{choi2012fast}) learning from several domains parallelly thus delaying the model selection decision to the decoding process. 

Next, utilization of domain-specific lexicons is another way of adapting taggers, as they can improve tagging performance significantly \cite{coden2005domain,ruch2000minimal}. 
Some studies extend existing \acrshort{pos} dictionaries \cite{divita2006dtagger}, while others build new ones \cite{Smith2006}. 
In brief, all such experiments yield significantly reduced error rates. 

Concerning tagging algorithms, researchers tend to prefer already existing applications. 
One of the most popular system is the OpenNLP toolkit \cite{Baldridge2002}, which is e.g. the basis of the cTakes system \cite{Savova2010}.
Further on, Brill’s method \cite{Brill1992} and TnT \cite{Brants2000} are widely used (e.g. \cite{hahn2004tagging,Savova2010,pestian2004development}) as well. 
Besides, other \acrshort{hmm}-based solutions were also shown to perform well \cite{barrett2011token,coden2005domain,divita2006dtagger,hahn2004tagging,pakhomov2006developing,rost2008lessons,ruch2000minimal} on biomedical texts. 

Moving on, a number of experiments have revealed \cite{ferraro2013improving,ruch2000minimal,Smith2006} that domain-specific \acrshort{oov} words are behind the reduced performance of taggers. 
Therefore, successful methods employ either guessing algorithms \cite{barrett2011token,divita2006dtagger,rost2008lessons,ruch2000minimal,Smith2006} or broad-coverage lexicons (as detailed above). 
Beyond supervised algorithms, other approaches were also shown to be effective: Miller et al. \cite{miller2007building} use semi-supervised methods;
%\footnote{This algorithm needs raw data from the target domain, while an annotated general corpus is still used.}; 
Dwinedi and Sukhadeve build a tagger based only on rules \cite{dwivedi8rule}; while Ruch et al. propose a hybrid system \cite{ruch2000minimal}. 
Further on, automatic domain adaptation methods (such as EasyAdapt \cite{daume2007frustratingly}, ClinAdapt \cite{ferraro2013improving} 
or reference distribution modelling  \cite{tateisi2006subdomain}) also perform well. As a drawback, they need an appropriate amount of manually annotated data from the target domain limiting their applicability. 

%First, we examine special properties of clinical notes, then a proper disambiguation methodology is being presented. 
Our method builds on a baseline tagging chain composed of a trigram tagger (introduced in Section \ref{sec:purepos}) and a broad coverage morphological analyzer. 
The latter tool employs a domain-adapted lexicon, while the tagger is adapted to the domain with further components.
%For this, an error analysis of the baseline system is used to adjust the output of the tagger to the domain.

\subsection{The clinical corpus}

As there is no corpus of clinical records available manually annotated with morphological analyses, a new one was created. 
These texts contain about 600 sentences extracted from notes of 24 different clinics. 
First, textual parts of the records were identified (as described in \cite{Siklosi2012}), then the paragraphs to be processed were selected randomly. 
After these, sentence boundary segmentation, tokenization and normalization was performed manually aided by methods of Section \ref{sec:clinical_segmentation}. 
Manual spelling correction was carried out using the system of Siklósi et al. \cite{Siklosi2013}. 
Finally, morphological disambiguation was performed: the initial annotation was provided by PurePos, then its output was corrected manually. 

As regards morphological annotation of texts, clinical notes have special properties differing from general Hungarian, which have been considered during their analysis. 
%Beside characteristics described above, the corpus has further specialties. 
These texts contain numerous \textit{x} tokens denoting multiplication, thus they are labeled as numerals. 
Latin words and abbreviations dominate sentences, which we decided to analyze regarding their meaning. 
For instance, \textit{o.} denotes \textit{szem} `eye’ thus it is tagged as a noun (\textsc{n.nom}). 
Further on, medicines brand names are common as well, which were almost always found to be singular nouns. 
Finally, numerous sentences lack final punctuation marks that are not recovered in the test corpus. 

The manually annotated corpus was split into two parts (cf. Table \ref{tab:clin_corpus}) for our experiments. 
The first one was employed for development purposes, while new methods were evaluated on the second part.
%

\begin{table}[H]
\centering
\caption{Size of the clinical corpus created}
\label{tab:clin_corpus}
\begin{tabular}{ l  r  r } 
\hline
& Sentences & Tokens \\
\hline
Development set & 240 & 2,230 \\
Test set & 333 & 3,155 \\
\hline
\end{tabular}
\end{table}


%Further on, special properties of clinical texts need to be considered. 
These records are created in a special environment, thus they differ from general Hungarian in several aspects (cf. \cite{Orosz2013a,Siklosi2013b,Siklosi2012}):

\begin{enumerate} %[\itshape a\upshape)]
 \item notes contain a lot of erroneously spelled words,
 \item sentences generally lack punctuation marks and sentence initial capitalization, 
 \item measurements are frequent and have plenty of different (erroneous) forms,
 \item a lot of (non-standard) abbreviations occur in such texts and
 \item numerous medical terms are used originating from Latin.
\end{enumerate}

\subsection{The baseline setting and its most common errors}
\label{sec:baseline}

We built a baseline chain and analyzed its errors to improve the overall annotation quality.
It uses the Humor analyzer, which produces \emph{(morpho-syntactic tag, lemma)} pairs as analyses. 
(The output of the \acrshort{ma} is extended with the new analysis of the \textit{x} token to fit the corpus to be tagged.)
Further on, analysis candidates are disambiguated by PurePos. 

This baseline tagger produces 86.61\% token accuracy\footnote{Precision is calculated considering correct full analyses of tokens, not counting punctuation marks.} on the development set, which is remarkably lower than tagging results for general Hungarian using the same components (96--98\% as in \cite{Orosz2013b,zsibrata2013magyarlanc}). 
Further on, sentence-based precision shows that less than the third (28.33\%) of the sentences were tagged correctly. 
This fact indicates that the models of the baseline algorithm alone are weak for this task. 
Therefore, we investigated the most common errors of the chain.

\begin{table}[H]
\centering
\caption{Distribution of the most frequent error types caused by the baseline algorithm (measured on the development set)}
\label{tab:error_types}
\begin{tabular}{ l r r } 
\hline
Class & Frequency & Ratio \\
\hline
Abbreviations and acronyms & 119 & 49.17\% \\
Out-of-vocabulary words & 66 & 27.27\% \\
Domain-specific PoS of word forms & 36 & 14.88\% \\
% Numbers & 0.02\% \\
% Other & 15 & 0.06\% \\
\hline
\end{tabular}
\end{table}

Table \ref{tab:error_types} shows that the top error class is composed of mistagged abbreviations and acronyms. 
A reason for this is that most of the abbreviated tokens are previously not seen by the tagger.
Therefore, their labels are produced by the tool's guesser module, which is not prepared for handling such tokens. 
What is more, these abbreviations usually refer to medical terms (and their inflected forms) originating from Latin, thus differing notably from standard ones.

Another class of mistakes was caused by \acrlong{oov} words. 
These are specific to the clinical domain and often originate from Latin.
Although this observation is in accordance with the \acrshort{pos} tagging results for medical English, listing of such terms' analyses is not a satisfactory solution to the problem, since the number of inflected forms is significantly larger compared to English. 

Finally, domain-specific usage of some words leads the tagger astray as well. 
An example is the class participles which are mislabeled as past tense verbs. 
E.g. \textit{javasolt} `suggested’  and  \textit{felírt} `written’ are common words in the corpus, but have different \acrshort{pos} tag distributions in this domain. 
Further on, several erroneous tags are due to the lexical ambiguity being present in Hungarian (such as \textit{szembe} which can refer to `into an eye’ or `toward/against’). 

%In sum, our investigation shows that most of the errors of the baseline system can be classified into the three categories shown in Table \ref{tab:error_types}. 
Based on the classification of errors above, domain-adaptation techniques were introduced enhancing the overall accuracy of the chain.

\subsection{Domain adaptation experiments}

%Systematic changes are carried out to improve the accuracy of the chain. 
%In doing so, each enhancement is evaluated against the development corpus to monitor their usefulness.

\subsubsection{Utilizing an extended morphological lexicon}
\label{sec:ma-extension}

Supervised tagging algorithms commonly use augmented lexicons reducing the number of out-of-vocabulary words (see Section \ref{sec:biomed_tag}). 
In the case of Hungarian, this must be performed at the level of the \acrlong{ma}, since inflection is a momentous phenomenon. 
Extension of the lexicon was carried out by Attila Novák \cite{Orosz2014} adding 40,000 different lemmata to the analyzer. 
For this, he used a spelling dictionary of medical terms \cite{Fabian1992} and a freely available list of medicines \cite{Foigazgatosag2012}.
By employing the enhanced lexicon, the ratio of \acrshort{oov} words was reduced to 26.19\% (from 34.57\%) that also improved the overall accuracy to 92.41\% (on the development set). 
Further on, the medical dictionary \cite{Fabian1992} used contained numerous abbreviated tokens as well, thus the usage of the augmented analyzer also helped to decrease the number of mistagged abbreviations.

\subsubsection{Dealing with acronyms and abbreviations}

Despite improvements above, numerous errors made by the enhanced tagger were still connected to abbreviations. 
Thus, we investigated the erroneous tags of abbreviated terms first, then methods were introduced for improving the performance of the disambiguation chain. 

A detailed examination revealed that some erroneous tags were due to the over-generating nature of Humor. 
To fix such problems, we applied a simple filtering method. 
An analysis of a word with an attached full stop was considered to be a false candidate if the lemma candidate is not an abbreviation. 
Consequently, the overall accuracy was increased notably, reducing the number of errors on the development set by 9.20\%.
%(cf. ``Filtering'' in Table \ref{tab:abbrev_fixes}).

Another typical error type was the mistagging of unknown acronyms. 
Since PurePos did not employ features  dealing with such cases, these tokens were usually left to the suffix guesser resulting in incorrect annotation. 
In addition, our investigation shows that acronyms should be tagged as singular nouns in most of the cases. 
To annotate them properly, a pattern matching component was developed relying on surface features.
% (see ``Acronyms'' in Table \ref{tab:abbrev_fixes}). 

Finally, the rest of the errors were connected to those abbreviations which were both unknown to the analyzer and had not been seen previously. 
Therefore, the abbreviations labels was compared to that of the Szeged Corpus (see Table \ref{tab:pos_distribution} below).
While there are common properties between the two datasets (such as the ratio of adverbs), discrepancies are more significant. 
The most important difference is the proportions of adjectives: it is notably higher in the medical domain than in general Hungarian. 
Moreover, these values are more expressive if we consider that 10.85\% of the tokens are abbreviated in the development set, while the same ratio is only 0.37\% in the Szeged Corpus. 

\begin{table}[H]
\centering
\caption{Morpho-syntactic tag frequencies of abbreviations on the development set}
\label{tab:pos_distribution}
\begin{tabular}{ l r r} 
\hline
Tag & Clinical texts & Szeged Corpus  \\ 
\hline
\scshape{n.nom} & 67.37\% & 78.18\% \\
\scshape{a.nom} & 19.07\% & 3.96\% \\
\scshape{conj} & 1.27\% & 0.50\% \\
\scshape{adv} & 10.17\% & 11.86\% \\
Other & 2.12\% & 5.50\% \\
\hline
\end{tabular}
\end{table}

Since the nominal noun tag is the most frequent amongst abbreviations, a plausible method (``UnkN'') was to assign the \textsc{n.nom} label to unknown ones. 
Meanwhile, we kept the original word forms as lemmata. 
Although this approach is rather simple, it resulted in a surprisingly high (31.54\%) error rate reduction (cf. Table \ref{tab:abbrev_fixes}). 

\begin{table}[H]
\centering
\caption{Accuracy scores of the abbreviation handling improvements on the development set}
\label{tab:abbrev_fixes}
\begin{tabular}{ l l r } 
\hline
ID & Method &  Accuracy \\
\hline
0 & Medical lexicon & 90.11\% \\
1 & 0 + Filtering & 91.02\% \\
2 & 1 + Acronyms & 91.41\% \\
3 & 2 + UnkN & \underline{94.12\%} \\
4 & 2 + UnkUni & 92.82\% \\
5 & 2 + UnkMLE & 94.01\% \\
\hline
\end{tabular}
\end{table}

Next, we tried to approximate the analyses of abbreviations with the distribution of tags observed in Table \ref{tab:pos_distribution}. 
First, we utilized (``UnkUni'') a uniform distribution over their labels. % using the development set.
The labels \textsc{a.nom}, \textsc{a.pro}, \textsc{adv}, \textsc{conj}, \textsc{n.nom}, \textsc{v.3sg} and \textsc{v.pst\_ptcl} were used with equal probability as a sort of guessing algorithm.

Beside these, another reasonable method was to employ \acrlong{mle} for calculating a priori probabilities of labels (``UnkMLE''). 
In that way, relative frequency estimates were computed for all the tags enlisted above.

Comparing the performance of these enhancements (cf. Table \ref{tab:abbrev_fixes}), we found that this approach can also increase the overall performance, but the simple ``UnkN'' performs the best.
This can be due to the fact that the data available could be insufficient for estimating probability distribution of labels properly.

\subsubsection{Choosing the proper training data}

Since many studies showed (cf. Section \ref{sec:biomed_tag}) that the training data set significantly affects the result of a data-driven annotation chain, we investigated sub-corpora of the Szeged Corpus. 
Several properties (cf. Table \ref{tab:subcorpora_attrib}) were examined\footnote{Measurements regarding the development set were calculated manually where it was necessary.} to find a decent domain to learn from for tagging clinical Hungarian. 

\begin{table}[H]
\centering
\caption{Comparing Szeged Corpus with clinical texts}
\label{tab:subcorpora_attrib}
\begin{tabular}{ l r r r r r } 
\hline
\multicolumn{1}{l}{\multirow{2}{*}{Corpus}} & \multicolumn{1}{c}{Avg. sent.} & \multicolumn{1}{c}{Abbrev.}  &  \multicolumn{1}{c}{Unknown} & \multicolumn{2}{c}{Perplexity} \\
 & \multicolumn{1}{c}{length} & \multicolumn{1}{c}{ratio} &  \multicolumn{1}{c}{ratio} & \multicolumn{1}{c}{Words} & \multicolumn{1}{c}{Tags} \\
\hline
Szeged Corpus & 16.82 & 0.37\%  & \underline{1.78\%} & 2318.02 & 22.56\\
\hspace{0.2cm} Fiction & 12.30 & 0.10\% & 2.44\% & 995.57 & 32.57\\
\hspace{0.2cm} Compositions & 13.22 & 0.14\% & 2.29\% & 1,335.90 & 30.78\\
\hspace{0.2cm} Computer & 20.75 & 0.14\% & 2.34\% & 854.11 & 22.89\\
\hspace{0.2cm} Newspaper & 21.05 & 0.20\% & 2.10\% & 1,284.89 & \underline{22.08}\\
\hspace{0.2cm} Law & 23.64 & 1.43\% & 2.74\% & \underline{824.42} & 29.79\\
\hspace{0.2cm} Short business news & 23.28 & 0.91\% & 2.50\% & 859.33 & 27.88\\
% \hline
Development set & 9.29 & 10.85\% & -- & -- & -- \\
\hline
\end{tabular}
\end{table}

First of all, an important attribute of texts is the length of sentences. 
Shorter sentences tend to have simpler grammatical structure, while longer ones are grammatically more complex. 
Further on, clinical texts have a vast amount of abbreviations, thus their ratio can also serve as a relevant metric. 
In addition, the accuracy of a tagging system depends on the ratio of unknown words heavily, therefore their proportions were calculated. 
For this, we measured the ratio of \acrshort{oov} words on the development set. 
%In doing so, the vocabularies of training sets are compared to the training corpora (see Table \ref{tab:subcorpora_attrib}). 

Perplexity was also computed, since it can measure similarities of texts \cite{kilgarriff1998measures}. 
The calculation was carried out as follows: trigram models of word and tag sequences were trained on each corpus using Kneser-Ney smoothing, then all of them were evaluated on the development set\footnote{We used the SRILM toolkit \cite{stolcke2002srilm} for training models and measuring perplexity.}.

Our examination shows that neither none of the parts of the Szeged Corpus contains as much abbreviated terms as clinical texts have. 
Likewise, sentences written by clinicians are significantly shorter than those of the Szeged Corpus. 
Neither the calculations above, nor the ratio of unknown words suggests using any of the  subcorpora for training. 
However, the perplexity scores contradict: sentences from the law domain share the most phrases with clinical notes, while news texts have the most similar grammatical structures. 

\begin{table}[H]
\centering
\caption{Evaluation of the tagger on the development set trained with domain-specific subcorpora of the Szeged Corpus}
\label{tab:eval_subcorpora}
\begin{tabular}{ l r } 
\hline
Corpus & Morph. disambiguation accuracy \\
\hline
Szeged Corpus & \underline{94.73\%} \\
\hspace{0.2cm} Fiction & 92.01\% \\
\hspace{0.2cm} Compositions & 91.97\% \\
\hspace{0.2cm} Computer & 92.73\% \\
\hspace{0.2cm} Newspaper & \underline{93.29\%} \\
\hspace{0.2cm} Law & 92.17\% \\
\hspace{0.2cm} Short business news & 92.69\% \\
\hline
\end{tabular}
\end{table}


Since similarity measurements were not in accordance with each other, all sub-corpora were tested as training data for tagging clinical texts. 
(These experiments were performed using the previously enhanced tagging chain.)
The accuracy scores of taggers (cf. Table  \ref{tab:eval_subcorpora}) on the development set show that training on a subcorpus cannot improve the performance. 
Therefore, we decided to use the whole Szeged Corpus to train our system.

\subsection{Evaluation}

The improved chain (cf. Table \ref{tab:improvements}) was evaluated by investigating the part-of-speech tagging, lemmatization and the whole morphological annotation performance.

\begin{table}[H]
\centering
\caption{Accuracy scores of the improved tagger on the test set}
\label{tab:improvements}
\begin{tabular}{ l l r r r} 
\hline
ID & Method & PoS tagging & Lemmatization & Morph. disambig. \\
\hline
0 & Baseline system & 90.57\% & 93.54\% & 88.09\% \\
1 & 0 + Lexicon extension & 93.89\% & 96.24\% & 92.41\% \\
2 & 1 + Handling abbreviations & \underline{94.81\%} & \underline{97.60\%} & \underline{93.73\%} \\
%3 & 2 + Training data selection & 94.25\% & 97.36\% & 93.29\% \\
\hline
\end{tabular}
\end{table}

First of all, results show that the baseline method annotated almost 12\% of the tokens erroneously, while our enhancements raised the ceiling of the full morphological tagging accuracy to 93.73\%.
Therefore, we managed to eliminate almost half (47.36\%) of the errors. 
Next, precision scores also indicate that the error rate reduction is mainly due to the extended lexicon.
However, the better handling of abbreviations also increased the performance significantly (Wilcoxon test of paired samples, p < 0.05).
Therefore, our improvements yielded a system having satisfactory performance for morphologically parsing clinical texts.


This study revealed that abbreviations and \acrlong{oov} words cause the most of the errors for tagging Hungarian clinical texts.
We introduced numerous enhancements dealing with them, although not all of them were successful.
This could be due to the small amount of annotated data used inhibiting the better modeling of the domain.
%Further on, Hungarian is a less-resource language and as such we did not manage to find a better (sub)corpus for tagging clinical Hungarian than the whole Szeged Corpus. 


