
\section{Introduction}

Hospitals produce a huge amount of clinical notes that have been solely used for archiving purposes and have generally been inaccessible to researchers. 
However, application of recent \acrshort{nlp} technology can make accessible the hidden knowledge of archived records, thus boosting medical research. 
While developing text processing tools for medicians is an emerging field in many developed countries, this is not true for less-resourced languages.

To be able to extract information from medical texts, they must be preprocessed properly. 
Firstly, adequate text segmentation methods 
%\footnote{While the term \emph{text segmentation} is widely used for diverse tasks, in our work it denotes the process of dividing text into tokens and sentences.} 
are required for finding token and sentence boundaries. 
Secondly, morphological tagging is an indispensable step for information extraction scenarios. 
%Further on, normalization and spelling-error correction can be also necessary depending on corpora.

Considering the case of Hungarian, there are only a few studies on processing medical records. 
Recently, Siklósi et al. \cite{Siklosi2012,Siklosi2013} have presented a system that is able to correct spelling errors in clinical notes. 
Their system uses a mixture of language models to generate correction candidates, however it focuses only on correctly segmented words. 
Next, an abbreviation resolution method has also been presented by them \cite{Siklosi2013b}, however, problems of text segmentation and  morphosyntactic tagging are still untouched. 
As far as we know, no study investigates such preprocessing tasks in the clinical Hungarian domain. 

This chapter presents accurate preprocessing algorithms for noisy medical texts.
Results are developed and presented for only Hungarian, but they are designed in a way to perform well on other morphologically complex languages as well. 
Firstly, an effective method is introduced for detecting sentence and token boundaries.
The presented system builds on well-known tokenization rules boosting them with the knowledge of a morphological analyzer and the output of an unsupervised filtering algorithm.
Secondly, tagging experiments are presented yielding a viable morphological tagger for clinical Hungarian. 
The proposed tool builds on PurePos by fixing its most common errors in the domain.

\section{Text segmentation in electronic health records}\label{sec:clin_segm}

Error propagation in a text-processing chain is usually a notable problem, therefore accurate text segmentation methods are essential to process any sort of text properly.
However, notes written by doctors are extremely noisy containing errors which inhibit applications of existing tools.

Even though existing segmentation methods for Hungarian performing well on general domains, they have serious difficulties on clinical texts.
These originate in special properties of the text involving
\begin{inparaenum}[\itshape a\upshape)]
 \item typing errors (i.e. mistyped tokens, nonexistent strings of falsely concatenated words) and
 \item nonstandard usage of Hungarian.
\end{inparaenum}
While errors of the first type can generally be corrected with a rule-based tool, others need advanced methods. 

In this section, a hybrid approach to normalization and segmentation of Hungarian clinical records is presented. 
The method consists of three phases: first, a rule-based clean-up step is performed; then tokens are partially segmented; finally, sentence boundaries are determined. 
We start by detailing the tool's architecture. 
Then, key elements of the sentence boundary detection (SBD) algorithm are described. 
Finally, the system presented is evaluated against a gold standard corpus, and is compared to other tools available.

\subsection{Previous work on text segmentation}

Even though, numerous approaches deal with English noisy texts, only a few attempts have been made (cf. \cite{Siklosi2012,Siklosi2013,Siklosi2013b}) for Hungarian. In addition, the latter studies completely ignores the segmentation problem. What is more, most of the attempt attempts for clinical English also ignores the description of the tokenization/SBD algorithm applied. Therefore, we start with reviewing general segmentation techniques then continue on describing which of them are used successfully for the biomedical domain.

The task of text segmentation is often composed of several subtasks: normalization of noisy text (when necessary), segmentation of words, and sentence boundary detection.  
Although, these subtasks are generally performed one after another, there are approaches (e.g. \cite{zhu2007unified}), where text segmentation and normalization are treated as a unified tagging problem. Further on, handling of abbreviations is often involved during the process, since their identification helps detecting sentence and token boundaries.

As regards tokenization, it is generally treated as a simple engineering problem\footnote{In the case of alphabetic writing systems.} aiming to split punctuation marks from word forms. On the contrary, SBD is a rather researched topic. As Read et al. summarize \cite{read2012sentence}, sentence segmentation approaches fall into three classes: 
\begin{enumerate}
 \item rule-based methods employing domain- or language-specific knowledge (such as abbreviations); 
 \item supervised machine learning approaches, which may not be robust amongst domains (being specialist on the training corpus); 
 \item unsupervised learning methods, which extract their knowledge from raw unannotated data. 
\end{enumerate}

As regards ML attempts, one of the first methods was presented by Riley \cite{riley1989some} applying decision-tree learners for disambiguating full stops. He utilized mainly lexical features, such as word length and case, and probabilities of a word being sentence-initial or sentence-final. 
Next, the SATZ system of Palmer et al. \cite{palmer1997adaptive} employs machine learning algorithms employing contextual and PoS features as well. 
Since it can be easily adapted adjusting the features, this system has been successfully applied \cite{palmer1997adaptive} to several European languages. 
Further on, the maximum entropy learning approach was also utilized by Reynar and Ratnaparkhi \cite{reynar1997maximum}. 
Their system classifies tokens containing `.', `?' or `!' by using contextual features and a prepared abbreviation list. A similar approach for English has been presented recently by
Recently, Gillick \cite{gillick2009sentence}. Their system uses support vector machines resulting in a state-of-the-art performance.

Beside machine learning, rule-based methods are commonly applied for the tasks. E.g. Mikheev presents \cite{mikheev2002periods} a small set of rules for detecting sentence boundaries (SB) with a high accuracy. 
In another system presented by him \cite{mikheev2000tagging}, the latter method is integrated into a PoS-tagging framework. This enhancement enables the classification of punctuation marks labeling them either as a sentence boundary, an abbreviation or both. 
Moving on, Kiss and Strunk have presented an unsupervised segmentation method recently. 
Their system, Punkt \cite{kiss2006unsupervised} uses scaled log-likelihood ratio for deciding whether a \emph{(word, period)} pair is a collocation or not.

Although tokenization and SBD tasks are well established fields in natural language processing, there are only a few attempts aiming medical texts. 
Sentence segmentation attempts for clinical texts fall into two classes: some develop rule-based systems (e.g. \cite{XuSDJWD10}), while most of the studies employ supervised machine learning algorithms (such as \cite{apostolova2009automatic,cho2002text,Savova2010,taira2001automatic,tomanek2007sentence}).
These approaches usually train maximum entropy or CRF learners, thus large handcrafted training corpora are essential. Training data used is either domain-specific or general. 
In practice, training material from a related domain yields better performance. However Tomanek et al. argue \cite{tomanek2006reappraisal} argue on using general newswire texts,
%TODO: newswire vajon igaz?
showing that the domain of the training corpus is not critical.

As regards Hungarian, there are two text segmentation tools available. Huntoken \cite{halacsy2004creating} is an open source tool based on Mikheev’s rule-based system, while \texttt{magyarlanc} \cite{zsibrata2013magyarlanc} has an adapted version of MorphAdorner’s rule-based tokenizer \cite{kumar2009monk} and sentence splitter. Both of them are general-purpose employing rules and dictionaries.


Since we found that existing tools cannot process medical texts properly, this section present a study on segmenting texts of noisy clinical notes. First, we investigate special properties of the target by creating a manually segmented corpus. Considering the target data we present a method which combines high precision rules with an unsupervised SBD method with well established recall.

\subsection{Clinical text used}
\label{sec:clin_corpus}

A gold standard corpus is collected and manually corrected enabling the investigation and evaluation of segmentation approaches.
This process involved several steps. First, input texts had to be normalized first, since the data collected is highly erroneous.
We had to deal with the following errors\footnote{Text normalization is performed using regular expressions.}:
\begin{enumerate}
 \item doubly converted characters, such as `\&amp;gt;',
 \item typewriter problems (e.g. `1' and `0' is written as `l' and `o'),
 \item dates and date intervals that were in various formats with or without necessary whitespaces (e.g. `2009.11.11', `06.01.08'),
 \item missing whitespaces between tokens that usually introduced various types of errors, such as:
 \begin{enumerate}
  \item measurements 
  were erroneously attached to quantities (e.g. `0.12mg'),
  \item lack of whitespace around punctuation marks (e.g. `töröközegek.Fundus:ép.'),
 \end{enumerate}
 \item various formulation of numerical expressions.
\end{enumerate}
 
In order to investigate possible pitfalls of the algorithm being developed, the gold standard data was split into two sets of equal sizes: a development and a test set containing 1320 and 1310 sentences respectively. 
The first part is used to identify typical problems in the corpus and to develop the segmentation methods. The second part is used to verify our results. 

The distribution of abbreviations, punctuation marks and capitalization in a certain text help to reveal the unique segmentation problems of that documents. Therefore a comparison of clinical texts and a corpus of general Hungarian (Szeged Corpus \cite{Csendes2004}) is carried out: 
\begin{enumerate}
 \item 2.68\% of tokens found in clinical corpus sample are abbreviations while the same ratio for general Hungarian is only 0.23\%; 
 \item sentences taken from the Szeged Corpus almost always end in a sentence final punctuation mark (98.96\%), while these are totally missing from clinical statements in 48.28\% of the cases; 
 \item sentence-initial capitalization is a general rule in Hungarian (99.58\% of the sentences are formulated properly in the Szeged Corpus), but its usage is not common in the case of clinicians (12.81\% of the sentences start with a word that is not capitalized); 
 \item the amount of numerical data is significant in medical records (13.50\% of sentences consist exclusively of measurement data and abbreviations), while text taken from the general domain rarely contains statements that are full of measurements. 
\end{enumerate}

\subsection{Evaluation metrics}

There are no unified metric being commonly used for text segmentation tasks.
Researchers specializing in machine learning approaches prefer to calculate precision, recall and $F$-measure.
Others, having a  background in speech recognition, prefer to compute NIST and Word Error Rate. 
Recently, Read et al. have reviewed \cite{read2012sentence} the current state-of-the-art in sentence boundary detection proposing a unified metric for comparing the performance of different approaches. 
Their method allows to measure sentence boundaries at any position, since characters are labeled as sentence-finals or non sentence-finals. In doing so, simple accuracy can be used as a metric. 

Our work rely on the metric introduced by Read et al. generalizing it for our task. The text is considered as a sequence of characters and empty strings. Segmentation is treated as a single classification problem, where all of the entities (either characters or empty string between them) are labeled with the following tags: 
\begin{description}
 \item[$\langle$T$\rangle$] --  if the entity is a token boundary,
 \item[$\langle$S$\rangle$] -- if it is a sentence boundary,
 \item[$\langle$None$\rangle$] -- if the entity is neither.
\end{description}
In doing so, this classification scheme enables us to calculate accuracy for the unified task. 
Further on, it is important to measure the subsystem's performance, thus precision and recall values are calculated for both word tokenization and SBD. 
Precision becomes more important than recall for segmenting sentences. It is because
an erroneously split sentence may cause information loss\label{sec:loss}, while statements might still be extracted from multi-sentence text. 
Consequently, $F_{0.5}$ is computer for SBD, while word tokenization is evaluated with the standard $F_1$ measure. 

\subsection{Segmentation Methods}
\label{sec:clinical_segmentation}

Our system is built up from several components. 
First, we introduce a baseline rule-based method which is used for marking word and sentence boundaries\footnote{Rules and heuristics used are formulated investigating the development corpus.}. 
Next, we detail its extensions yielding better segmentation algorithms. %TODO: esetleg egy ábra elfér még.

\subsubsection{Baseline word tokenization and sentence segmentation}

The baseline rule-based method is composed of two parts. First, it tokenizes words (BWT) by using regular expressions implemented in standard tokenizers. This module does not try disambiguate periods attached to the ends of words, since proper handling of such words would need to recognize abbreviations properly. 

Sentence segmentation (BSBD) in a subsequent component being performed in a way to avoid information loss (as described in section \ref{sec:loss}.) 
In doing so, the method minimizes the possibility of making false-positive errors by splitting sentences only if there is a high confidence of success. 
These cases are when:
\begin{enumerate} 
 \item a period or exclamation mark directly follows another punctuation mark token\footnote{Question marks are not considered as sentence-final punctuation marks, since they generally indicate a questionable finding in clinical texts.};
 \item a line starts with a full date, and is followed by other words (The last white-space character before the date is marked as SB.);
 \item a line begins with the name of an examination followed by a semicolon and a sequence of measurements.
\end{enumerate}

Implementing these simple observation yields 100\% precision and 73.38\% recall for the token segmentation task on the development set. The corresponding values for the sentence boundary detection are 98.48\% and 42.60\% respectively. 
Results on the development set indicate that less than half of the sentence boundaries are discovered, thus reveal the need of further improvement.
Further on, an analysis of errors also unfolds that the tokenization module has difficulties only with sentence final periods. These sort of errors are the effects of the conservative tokenization algorithm, since words with punctuation mark attached are left ambiguous.
Therefore, this investigation implies that an advanced sentence boundary detection algorithm is necessary for achieving higher recall scores.

\subsubsection{Improvements on sentence boundary detection}\label{sec:improvements}

To improve SBD results of the baseline method we investigate the applicability of common techniques. There are two sort of indicators generally used for detecting sentence boundaries: 
\begin{description}
 \item[period] when a period ($\bullet$) is attached to a word a sentence boundary is found for sure only if the token is not an abbreviation;
 \item[capitalization] if a word starts with a capital letter and it is neither part of a proper name nor of an acronym.
\end{description}

Unfortunately, these are not directly applicable in our case. First of all, medical abbreviations are too diverse: clinicians usually introduce new ones not being part of the standard. 
Further on, Latin words and abbreviations are sometimes capitalized by mistake. In addition, some subclauses begin with capitalized words as well. 
Finally, as shown in Section \ref{sec:clin_corpus}, several sentence boundaries lack both of these indicators.

Even though these features are not proper indicators they can still be for used finding sentences. In addition, one does not need a full list of possible abbreviations neither. It is enough to find \emph{evidence} for the separateness of a word and the period attached to classify this position as a sentence boundary. 
This idea was introduced by Kiss and Strunk \cite{kiss2006unsupervised} and is being adapted in this study.

Scale-log likelihood ratio was originally used for identifying collocations by Dunning \cite{dunning1993accurate}, however it has been adapted for the sentence segmentation task in Punkt. The basic idea used is to handle abbreviations as collocations of words and periods. In practice, this is formulated via a null hypothesis \eqref{eq:h0} and an alternative one \eqref{eq:ha}. 

\begin{equation} \label{eq:h0}
H_0: P(\bullet|w) = p = P(\bullet|\neg w)
\end{equation}
\vspace{-0.5cm}
\begin{equation} \label{eq:ha}
H_A: P(\bullet|w) = p_1 \neq p_2 = P(\bullet|\neg w) 
\end{equation}
\vspace{-0.5cm}
\begin{equation} \label{eq:loglambda}
log \lambda = -2 log \frac{L(H_0)}{L(H_A)}
\end{equation}


\eqref{eq:h0} expresses the independence of a \emph{(word, $\bullet$)} pair, while \eqref{eq:ha} formulates that their co-occurrence is not just by chance. $log \lambda$ score  \eqref{eq:loglambda} computes their ratio in a way to be asymptotically $\chi^2$ distributed. 
Therefore, it can be applied as a statistical test \cite{dunning1993accurate}. 
Kiss and Strunk found that pure $log \lambda$ score performs poorly\footnote{In terms of precision.} in abbreviation detection scenarios, thus they introduced scaling factors \cite{kiss2006unsupervised}. 
In doing so, their method loses the asymptotic relation to $\chi^2$ and becomes a simple filtering algorithm.

Utilizing their ideas we improved the original method in numerous places. 
First of all, the inverse of $\log\lambda$:$iscore=1/log\lambda$ is calculated as a base, since our goal is to find candidates co-occurring only by chance. In addition, we adapt existing scaling factors and introduce new ones to match the characteristics of the data.

The first scaling factor found in the original work \cite{kiss2006unsupervised} cannot be directly applied, since counts and count ratios do not indicate properly whether a token and the period is related in a clinical record. The reason behind this is that several sort of abbreviations with relative low frequencies. 
Next, length of words ($len$) has been shown to be a good indicator of abbreviations, since shorter tokens tend to be abbreviations, while longer ones do not. Thus, we reformulate the original function to penalize short words and reward longer ones. 
Having a medical abbreviation list of almost 200 \label{sec:abbrev} elements\footnote{The list is gathered with an automatic algorithm on the development corpus using word shape properties and frequencies. The most frequent elements are manually verified and corrected.} 
we found that that more than 90\% of the abbreviations are shorter than three characters. This fact led us to formulate the scaling factor in equation \eqref{eq:s_l}. 
In doing so, this modification can also decrease a score of candidate in contrast with the original formula in \cite{}.


\begin{equation} \label{eq:s_l}
S_{length}(iscore)= iscore \cdot \exp{(len/3-1)}
\end{equation}

Recently, HuMor \cite{Proszeky1994,Proszeky2005}  has been extended with the content of a medical dictionary \cite{Orosz:mszny2013}, thus this tool is used to enhance the sentence segmentation algorithm.  
Since the analyzer is able indicate whether an analyses refers to an abbreviation, its output is utilized by an indicator function \eqref{eq:sign}.
Furthermore, morphological lexicons are usually well-established resources, therefore applications can rely on them without any doubt. Consequently, a more confident factor is formulated (Equation \eqref{eq:s_m}) using larger weights compared to others. In doing so, the score is raised in case of full words, it is decreased for abbreviations, while values of unknown words are left as they were.

\begin{equation}\label{eq:sign}
 indicator_{morph}(word) =
  \begin{cases}
   1  & \text{if $word$ has an analysis of a known full word} \\
   -1 & \text{if $word$ has an analysis of a known abbreviation} \\
   0  & \text{otherwise}
  \end{cases}
\end{equation}

\begin{equation} \label{eq:s_m}
S_{morph}(iscore)= iscore \cdot \exp{( indicator_{morph} \cdot len^2)}
\end{equation}

Besides, another indicator has been found to fit well to the development data. Since, hyphens are generally not present in abbreviations but rather occurs in full words, the overall score is modified involving this observation. Thus, Equation \eqref{eq:s_h} is used raise the score of tokens having hyphens. For this $indicator_{hyphen}$ is utilized outputting $1$ only if the word contains a hyphen. 


\begin{equation} \label{eq:s_h}
S_{hyphen}(iscore)= iscore \cdot \exp{(indicator_{hyphen} \cdot len)}
\end{equation}

\begin{equation}
S = S_{length} \circ S_{morph} \circ S_{hyphen}
\end{equation}

Scaled $S(iscore)$ is calculated for all \emph{(word, $\bullet$)} pairs not followed by another punctuation mark. If this value is higher than a threshold, the period is regarded as a sentence boundary and it is detached.\footnote{Threshold value is empirically set to $1.5$.}
Investigating the scaled $\log \lambda$ performance, it is pipelined after the BSBD module resulting in 77.14\% recall and 97.10\% precision on the development set. These values indicates significant improvement but shows that many sentence boundaries are still not found.

To further improve the method word capitalization is utilized as well. A subsequent rule-based component is created dealing with capitalized words. 
Good SB candidates of these are the ones not following a non sentence terminating\footnote{Sentence terminating punctuation marks are the period and the exclamation mark for this task.} punctuation, and are not part of a named entity. 
Therefore, sequences of capitalized words are considered to be named entities and omitted as a first step. Then the rest of the candidates are processed with the morphological analyzer employing a simple heuristic for detecting sentence boundaries. If a word does not have a proper noun analysis but is capitalized it is marked as the beginning of a sentence.  
Investigating the component enhancement over BSBD on the development set we found that this module results in an increased recall (65.46\%) while keeps precision high (96.37\%). 

\subsection{Evaluation}

% Our hybrid algorithm has been developed using the development set, thus it is evaluated against the rest of the data. 
% %As the starting point of our comparison, we present the accuracy values of the preprocessed input text and the baseline tokenized one. 

\begin{table}[h]
\centering
\caption{Accuracy of the input text and the baseline segmented one}
\label{tab:base}
\begin{tabular}{ l  r } 
\hline
& Accuracy \\ 
\hline
Raw corpus  & 97.55\% \\
BSBD & 99.11\% \\
+ LLR & 99.72\% \\
+ CAP & 99.26\% \\
+ LLR + CAP & 99.74\% \\
\hline
\end{tabular}
\end{table}

Accuracy values in Table \ref{tab:base} measures the tool's performance on the overall segmentation task. 
All of the components are evaluated and compared to the baseline module and the raw preprocessed corpus.
The unsupervised SBD algorithm is marked with \emph{LLR}\footnote{Referring to the term log-likelihood ratio.}, while the second component is indicated by the term \emph{CAP}.
This metric is not a well balanced, since its values are relatively high even for the preprocessed text. Therefore we present individual improvement scores as well. 
Relative error rate reduction scores are provided in Table \ref{tab:reduction}. These values are calculated over the baseline method (BSBD) for each component and their collaboration as well. 

\begin{table}[h]
\centering
\caption{Error rate reduction over the accuracy of the baseline method}
\label{tab:reduction}
\begin{tabular}{ l  r } 
\hline
& Error rate reduction\\
\hline
LLR & 58.62\% \\
CAP & 9.25\% \\
LLR + CAP & 65.50\% \\
\hline
\end{tabular}
\end{table}


Considering sentence boundaries, a more detailed analysis is got by computing precision, recall and $F_{0.5}$ values in Table \ref{tab:prec_rec}. These results show that each component significantly increases the recall, while precision is just barely decreased. All in all, the combined hybrid algorithm\footnote{It is the composition of the BWT, BSBD, LLR and CAP components.} brings significant improvement over the well-established baseline.

\begin{table}[h]
\centering
\caption{Evaluation of the proposed sentence segmentation algorithm compared with the baseline}
\label{tab:prec_rec}
\begin{tabular}{ l r r  r  } 
\hline
& Precision & Recall & $F_{0.5}$ \\
\hline
Baseline & 96.57\% & 50.26\% & 81.54\%  \\
+ LLR & 95.19\% & 78.19\% & 91.22\% \\
+ CAP & 94.60\% & 71.56\% & 88.88\% \\
+ LLR + CAP & 93.28\% & 86.73\% & \underline{91.89\%} \\
\hline
\end{tabular}
\end{table}


While our approach has a focus on the sentence segmentation task, we show that it improves word tokenization as well. Table \ref{tab:tok_eval} presents measurements on word tokenization showing that our enhancements results in a higher recall, while they does not significantly decrease precision. \label{sec:eval}

\begin{table}[h]
\centering
\caption{Comparing tokenization performance of the new tool with the baseline one}
\label{tab:tok_eval}
\begin{tabular}{ l r r r} 
\hline
& Precision & Recall & $F_{1}$ \\
\hline
Baseline & 99.74\% & 74.94\% & 85.58\%  \\
Hybrid system & 98.54\% & 95.32\% & \underline{96.90\%} \\
\hline
\end{tabular}
\end{table}

Besides, we compare our method with freely available tools as well.
There are only two application aiming Hungarian text segmentation: are \texttt{magyarlanc} Huntoken.
The latter can be slightly adapted to a new domain by providing a set of abbreviations, thus two versions of it are evaluated. 
The first employs a set of general Hungarian abbreviations (\emph{HTG}), while the second utilizes an extended dictionary\footnote{As described in section \ref{sec:abbrev}.} containing medical ones as well(\emph{HTM}). 
Further on, Punkt \cite{kiss2006unsupervised} is involved in our comparision as well as the OpenNLP \cite{Baldridge2002} toolkit. The latter tool is a general framework building on maximum entropy methods, thus it can be applied to detect sentence boundaries as it is presented in \cite{reynar1997maximum}. For this we used the general-purpose Szeged Corpus as a training material. 

\begin{table}[h]
\centering
\caption{Comparision of the proposed hybrid SBD method with competing ones}
\label{tab:comparison}
\begin{tabular}{ l r r r} 
\hline
& Precision & Recall & $F_{0.5}$ \\
\hline
magyarlanc & 72.59\% & 77.68\% & 73.55\% \\
HTG & 44.73\% & 49.23\% & 45.56\% \\
HTM & 43.19\% & 42.09\% & 42.97\% \\
Punkt & 58.78\% & 45.66\% & 55.59\%  \\
OpenNLP & 52.10\% & 96.30\% & 57.37\% \\
Hybrid system & 93.28\% & 86.73\% & \underline{91.89\%} \\
\hline
\end{tabular}
\end{table}

First, results in Table \ref{tab:comparison} indicates that general segmentation methods fail on Hungarian clinical records in contrast to our new method. The hybrid approach presented bears with high precision and recall providing accurate sentence boundaries.
It has been found that the maxent approach has high recall as well, but boundaries marked by it are false positives in almost half of the cases. 
Further on, rules of \texttt{magyarlanc} seem to be robust, but the overall low performance inhibits its application for clinical texts. 
Finally, other tools do not just provide low recalls, but their precision values are around 50\% being too low for practical purposes. 

All in all, the presented segmentation method successfully deals with several sorts of imperfect sentence and word boundaries.
It performs better in terms of precision and recall than competing ones achieving a 92\% of $F_{0.5}$-score. Our results indicates that the the new hybrid algorithm is a proper tool for processing clinical Hungarian.




\pagebreak

\section{Morphological tagging of clinical notes}\label{sec:clin_tag}
Beside text segmentation, morphological tagging is also an indispensable task for information extraction scenarios. 
Even though tagging of general texts is well-known and considered to be solved, medical texts pose new challenges to researchers. 
In addition, English has been the main target of many studies investigating the biomedical domain up to the present time. 
Furthermore, there are just a few approaches for non- English data, neglecting agglutinative languages and particularly Hungarian.

This section investigates the tagging of clinical Hungarian by adapting existing methods. % which results in an accurate tagger.
Our work is structured as follows. 
Related studies are described first, then a corpus of clinical notes is presented.
%Latter texts has been collected for developing and evaluating tagging algorithms. 
%In Section \ref{sec:baseline}, we introduce a baseline morphological disambiguation chain and a detailed error analysis. 
Finally, domain adaptation enhancements are introduced which are then evaluated on the test corpus.

\subsection{Background}
\label{sec:biomed_tag}

In general, tagging of biomedical texts has an extensive literature, since numerous resources are accessible for English. 
On the contrary, much less manually annotated corpora of clinical texts are available. 
Further on, most of the work in this field was done for English, while only a few attempts were published for morphologically rich languages (e.g. \cite{oleynik2009performance,rost2008lessons}).

First of all, a common approach for tagging biomedical text is to train supervised sequence-classifiers. 
However, a drawback of these methods is that they require manually annotated texts which are hard to create. %labor-intense human work. 
Considering the types of training material, domain-specific corpora are used either alone \cite{pakhomov2006developing,Savova2010,Smith2006} or in conjunction with a (sub)corpus of general English \cite{coden2005domain,ferraro2013improving,miller2007building}. % as training data. 
While utilizing texts only from the target domain yields acceptable performance \cite{pakhomov2006developing,Savova2010,Smith2006}, 
several experiments have shown that accuracy further increases with incorporating annotated sentences from the general domain as well \cite{barrett2011token,coden2005domain}. 
It is shown (e.g. \cite{pestian2004development}) that the more data is used from the reference domain, the higher accuracy can be achieved. 
However, Hahn and Wermter argue for training learners only on general corpora \cite{hahn2004tagging} (for German). 
Besides, there are studies on automatic selection of the training data (e. g. \cite{liu2007heuristic}). % to increase accuracy. 
What is more, there are algorithms (such as \cite{choi2012fast}) learning from several domains parallelly thus delaying the model selection decision to the decoding process. 

Next, utilization of domain-specific lexicons is another way of adapting taggers, as they can improve tagging performance significantly \cite{coden2005domain,ruch2000minimal}. 
Some studies extend existing \acrshort{pos} dictionaries \cite{divita2006dtagger}, while others build new ones \cite{Smith2006}. 
In brief, all such experiments yield significantly reduced error rates. 

Concerning tagging algorithms, researchers tend to prefer already existing applications. 
One of the most popular system is the OpenNLP toolkit \cite{Baldridge2002}, which is e.g. the basis of the cTakes system \cite{Savova2010}.
Further on, Brill’s method \cite{Brill1992} and TnT \cite{Brants2000} are widely used (e.g. \cite{hahn2004tagging,Savova2010,pestian2004development}) as well. 
Besides, other \acrshort{hmm}-based solutions were also shown to perform well \cite{barrett2011token,coden2005domain,divita2006dtagger,hahn2004tagging,pakhomov2006developing,rost2008lessons,ruch2000minimal} on biomedical texts. 

Moving on, a number of experiments have revealed \cite{ferraro2013improving,ruch2000minimal,Smith2006} that domain-specific \acrshort{oov} words are behind the reduced performance of taggers. 
Therefore, successful methods employ either guessing algorithms \cite{barrett2011token,divita2006dtagger,rost2008lessons,ruch2000minimal,Smith2006} or broad-coverage lexicons (as detailed above). 
Beyond supervised algorithms, other approaches were also shown to be effective: Miller et al. \cite{miller2007building} use semi-supervised methods;
%\footnote{This algorithm needs raw data from the target domain, while an annotated general corpus is still used.}; 
Dwinedi and Sukhadeve build a tagger based only on rules \cite{dwivedi8rule}; while Ruch et al. propose a hybrid system \cite{ruch2000minimal}. 
Further on, automatic domain adaptation methods (such as EasyAdapt \cite{daume2007frustratingly}, ClinAdapt \cite{ferraro2013improving} 
or reference distribution modelling  \cite{tateisi2006subdomain}) also perform well. As a drawback, they need an appropriate amount of manually annotated data from the target domain limiting their applicability. 

%First, we examine special properties of clinical notes, then a proper disambiguation methodology is being presented. 
Our method builds on a baseline tagging chain composed of a trigram tagger (introduced in Section \ref{sec:purepos}) and a broad coverage morphological analyzer. 
The latter tool employs a domain-adapted lexicon, while the tagger is adapted to the domain with further components.
%For this, an error analysis of the baseline system is used to adjust the output of the tagger to the domain.

\subsection{The clinical corpus}

As there is no corpus of clinical records available manually annotated with morphological analyses, a new one was created. 
These texts contain about 600 sentences extracted from notes of 24 different clinics. 
First, textual parts of the records were identified (as described in \cite{Siklosi2012}), then the paragraphs to be processed were selected randomly. 
After these, sentence boundary segmentation, tokenization and normalization was performed manually aided by methods of Section \ref{sec:clinical_segmentation}. 
Manual spelling correction was carried out using the system of Siklósi et al. \cite{Siklosi2013}. 
Finally, morphological disambiguation was performed: the initial annotation was provided by PurePos, then its output was corrected manually. 

As regards morphological annotation of texts, clinical notes have special properties differing from general Hungarian, which have been considered during their analysis. 
%Beside characteristics described above, the corpus has further specialties. 
These texts contain numerous \textit{x} tokens denoting multiplication, thus they are labeled as numerals. 
Latin words and abbreviations dominate sentences, which we decided to analyze regarding their meaning. 
For instance, \textit{o.} denotes \textit{szem} `eye’ thus it is tagged as a noun (\textsc{n.nom}). 
Further on, medicines brand names are common as well, which were almost always found to be singular nouns. 
Finally, numerous sentences lack final punctuation marks that are not recovered in the test corpus. 

The manually annotated corpus was split into two parts (cf. Table \ref{tab:clin_corpus}) for our experiments. 
The first one was employed for development purposes, while new methods were evaluated on the second part.
%

\begin{table}[H]
\centering
\caption{Size of the clinical corpus created}
\label{tab:clin_corpus}
\begin{tabular}{ l  r  r } 
\hline
& Sentences & Tokens \\
\hline
Development set & 240 & 2,230 \\
Test set & 333 & 3,155 \\
\hline
\end{tabular}
\end{table}


%Further on, special properties of clinical texts need to be considered. 
These records are created in a special environment, thus they differ from general Hungarian in several aspects (cf. \cite{Orosz2013a,Siklosi2013b,Siklosi2012}):

\begin{enumerate} %[\itshape a\upshape)]
 \item notes contain a lot of erroneously spelled words,
 \item sentences generally lack punctuation marks and sentence initial capitalization, 
 \item measurements are frequent and have plenty of different (erroneous) forms,
 \item a lot of (non-standard) abbreviations occur in such texts and
 \item numerous medical terms are used originating from Latin.
\end{enumerate}

\subsection{The baseline setting and its most common errors}
\label{sec:baseline}

We built a baseline chain and analyzed its errors to improve the overall annotation quality.
It uses the Humor analyzer, which produces \emph{(morpho-syntactic tag, lemma)} pairs as analyses. 
(The output of the \acrshort{ma} is extended with the new analysis of the \textit{x} token to fit the corpus to be tagged.)
Further on, analysis candidates are disambiguated by PurePos. 

This baseline tagger produces 86.61\% token accuracy\footnote{Precision is calculated considering correct full analyses of tokens, not counting punctuation marks.} on the development set, which is remarkably lower than tagging results for general Hungarian using the same components (96--98\% as in \cite{Orosz2013b,zsibrata2013magyarlanc}). 
Further on, sentence-based precision shows that less than the third (28.33\%) of the sentences were tagged correctly. 
This fact indicates that the models of the baseline algorithm alone are weak for this task. 
Therefore, we investigated the most common errors of the chain.

\begin{table}[H]
\centering
\caption{Distribution of errors caused by the baseline algorithm -- dev. set}
\label{tab:error_types}
\begin{tabular}{ l r } 
\hline
Class & Frequency  \\
\hline
Abbreviations and acronyms & 49.17\% \\
Out-of-vocabulary words & 27.27\% \\
Domain-specific PoS of word forms & 14.88\% \\
% Numbers & 0.02\% \\
Other & 0.06\% \\
\hline
\end{tabular}
\end{table}

Table \ref{tab:error_types} shows that the top error class is composed of mistagged abbreviations and acronyms. 
A reason for this is that most of the abbreviated tokens are previously not seen by the tagger.
Therefore, their labels are produced by the tool's guesser module, which is not prepared for handling such tokens. 
What is more, these abbreviations usually refer to medical terms (and their inflected forms) originating from Latin, thus differing notably from standard ones.

Another class of mistakes was caused by \acrlong{oov} words. 
These are specific to the clinical domain and often originate from Latin.
Although this observation is in accordance with the \acrshort{pos} tagging results for medical English, listing of such terms' analyses is not a satisfactory solution to the problem, since the number of inflected forms is significantly larger compared to English. 

Finally, domain-specific usage of some words leads the tagger astray as well. 
An example is the class participles which are mislabeled as past tense verbs. 
E.g. \textit{javasolt} `suggested’  and  \textit{felírt} `written’ are common words in the corpus, but have different \acrshort{pos} tag distributions in this domain. 
Further on, several erroneous tags are due to the lexical ambiguity being present in Hungarian (such as \textit{szembe} which can refer to `into an eye’ or `toward/against’). 

%In sum, our investigation shows that most of the errors of the baseline system can be classified into the three categories shown in Table \ref{tab:error_types}. 
Based on the classification of errors above, domain-adaptation techniques were introduced enhancing the overall accuracy of the chain.

\subsection{Domain adaptation experiments}

%Systematic changes are carried out to improve the accuracy of the chain. 
%In doing so, each enhancement is evaluated against the development corpus to monitor their usefulness.

\subsubsection{Utilizing an extended morphological lexicon}
\label{sec:ma-extension}

Supervised tagging algorithms commonly use augmented lexicons reducing the number of out-of-vocabulary words (see Section \ref{sec:biomed_tag}). 
In the case of Hungarian, this must be performed at the level of the \acrlong{ma}, since inflection is a momentous phenomenon. 
Extension of the lexicon was carried out by Attila Novák \cite{Orosz2014} adding 40,000 different lemmata to the analyzer. 
For this, he used a spelling dictionary of medical terms \cite{Fabian1992} and a freely available list of medicines \cite{Foigazgatosag2012}.
By employing the enhanced lexicon, the ratio of \acrshort{oov} words was reduced to 26.19\% (from 34.57\%) that also improved the overall accuracy to 92.41\% (on the development set). 
Further on, the medical dictionary \cite{Fabian1992} used contained numerous abbreviated tokens as well, thus the usage of the augmented analyzer also helped to decrease the number of mistagged abbreviations.

\subsubsection{Dealing with acronyms and abbreviations}

Despite improvements above, numerous errors made by the enhanced tagger were still connected to abbreviations. 
Thus, we investigated the erroneous tags of abbreviated terms first, then methods were introduced for improving the performance of the disambiguation chain. 

A detailed examination revealed that some erroneous tags were due to the over-generating nature of Humor. 
To fix such problems, we applied a simple filtering method. 
An analysis of a word with an attached full stop was considered to be a false candidate if the lemma candidate is not an abbreviation. 
Consequently, the overall accuracy was increased notably, reducing the number of errors on the development set by 9.20\%.
%(cf. ``Filtering'' in Table \ref{tab:abbrev_fixes}).

Another typical error type was the mistagging of unknown acronyms. 
Since PurePos did not employ features  dealing with such cases, these tokens were usually left to the suffix guesser resulting in incorrect annotation. 
In addition, our investigation shows that acronyms should be tagged as singular nouns in most of the cases. 
To annotate them properly, a pattern matching component was developed relying on surface features.
% (see ``Acronyms'' in Table \ref{tab:abbrev_fixes}). 

Finally, the rest of the errors were connected to those abbreviations which were both unknown to the analyzer and had not been seen previously. 
Therefore, the abbreviations labels was compared to that of the Szeged Corpus (see Table \ref{tab:pos_distribution} below).
While there are common properties between the two datasets (such as the ratio of adverbs), discrepancies are more significant. 
The most important difference is the proportions of adjectives: it is notably higher in the medical domain than in general Hungarian. 
Moreover, these values are more expressive if we consider that 10.85\% of the tokens are abbreviated in the development set, while the same ratio is only 0.37\% in the Szeged Corpus. 

\begin{table}[H]
\centering
\caption{Morpho-syntactic tag frequencies of abbreviations -- dev. set}
\label{tab:pos_distribution}
\begin{tabular}{ l r r} 
\hline
Tag & Clinical texts & Szeged Corpus  \\ 
\hline
\scshape{n.nom} & 67.37\% & 78.18\% \\
\scshape{a.nom} & 19.07\% & 3.96\% \\
\scshape{conj} & 1.27\% & 0.50\% \\
\scshape{adv} & 10.17\% & 11.86\% \\
Other & 2.12\% & 5.50\% \\
\hline
\end{tabular}
\end{table}

Since the nominal noun tag is the most frequent amongst abbreviations, a plausible method (``UnkN'') was to assign the \textsc{n.nom} label to unknown ones. 
Meanwhile, we kept the original word forms as lemmata. 
Although this approach is rather simple, it resulted in a surprisingly high (31.54\%) error rate reduction (cf. Table \ref{tab:abbrev_fixes}). 

\begin{table}[H]
\centering
\caption{Comparison of the techniques aiming to handle acronyms and abbreviations --  dev. set}
\label{tab:abbrev_fixes}
\begin{tabular}{ l l r } 
\hline
ID & Method &  Precision \\
\hline
0 & Medical lexicon & 90.11\% \\
1 & 0 + Filtering & 91.02\% \\
2 & 1 + Acronyms & 91.41\% \\
3 & 2 + UnkN & \underline{94.12\%} \\
4 & 2 + UnkUni & 92.82\% \\
5 & 2 + UnkMLE & 94.01\% \\
\hline
\end{tabular}
\end{table}

Next, we tried to approximate the analyses of abbreviations with the distribution of tags observed in Table \ref{tab:pos_distribution}. 
First, we utilized (``UnkUni'') a uniform distribution over their labels. % using the development set.
The labels \textsc{a.nom}, \textsc{a.pro}, \textsc{adv}, \textsc{conj}, \textsc{n.nom}, \textsc{v.3sg} and \textsc{v.pst\_ptcl} were used with equal probability as a sort of guessing algorithm.

Beside these, another reasonable method was to employ \acrlong{mle} for calculating a priori probabilities of labels (``UnkMLE''). 
In that way, relative frequency estimates were computed for all the tags enlisted above.

Comparing the performance of these enhancements (cf. Table \ref{tab:abbrev_fixes}), we found that this approach can also increase the overall performance, but the simple ``UnkN'' performs the best.
This can be due to the fact that the data available could be insufficient for estimating probability distribution of labels properly.

\subsubsection{Choosing the proper training data}

Since many studies showed (cf. Section \ref{sec:biomed_tag}) that the training data set significantly affects the result of a data-driven annotation chain, we investigated sub-corpora of the Szeged Corpus. 
Several properties (cf. Table \ref{tab:subcorpora_attrib}) were examined\footnote{Measurements regarding the development set were calculated manually where it was necessary.} to find a decent domain to learn from for tagging clinical Hungarian. 

\begin{table}[H]
\centering
\caption{Comparing Szeged Corpus with clinical texts}
\label{tab:subcorpora_attrib}
\begin{tabular}{ l r r r r r } 
\hline
\multicolumn{1}{l}{\multirow{2}{*}{Corpus}} & \multicolumn{1}{c}{Avg. sent.} & \multicolumn{1}{c}{Abbrev.}  &  \multicolumn{1}{c}{Unknown} & \multicolumn{2}{c}{Perplexity} \\
 & \multicolumn{1}{c}{length} & \multicolumn{1}{c}{ratio} &  \multicolumn{1}{c}{ratio} & \multicolumn{1}{c}{Words} & \multicolumn{1}{c}{Tags} \\
\hline
Szeged Corpus & 16.82 & 0.37\%  & \underline{1.78\%} & 2318.02 & 22.56\\
\hspace{0.2cm} Fiction & 12.30 & 0.10\% & 2.44\% & 995.57 & 32.57\\
\hspace{0.2cm} Compositions & 13.22 & 0.14\% & 2.29\% & 1,335.90 & 30.78\\
\hspace{0.2cm} Computer & 20.75 & 0.14\% & 2.34\% & 854.11 & 22.89\\
\hspace{0.2cm} Newspaper & 21.05 & 0.20\% & 2.10\% & 1,284.89 & \underline{22.08}\\
\hspace{0.2cm} Law & 23.64 & 1.43\% & 2.74\% & \underline{824.42} & 29.79\\
\hspace{0.2cm} Short business news & 23.28 & 0.91\% & 2.50\% & 859.33 & 27.88\\
% \hline
Development set & 9.29 & 10.85\% & -- & -- & -- \\
\hline
\end{tabular}
\end{table}

First of all, an important attribute of texts is the length of sentences. 
Shorter sentences tend to have simpler grammatical structure, while longer ones are grammatically more complex. 
Further on, clinical texts have a vast amount of abbreviations, thus their ratio can also serve as a relevant metric. 
In addition, the accuracy of a tagging system depends on the ratio of unknown words heavily, therefore their proportions were calculated. 
For this, we measured the ratio of \acrshort{oov} words on the development set. 
%In doing so, the vocabularies of training sets are compared to the training corpora (see Table \ref{tab:subcorpora_attrib}). 

Perplexity was also computed, since it can measure similarities of texts \cite{kilgarriff1998measures}. 
The calculation was carried out as follows: trigram models of word and tag sequences were trained on each corpus using Kneser-Ney smoothing, then all of them were evaluated on the development set\footnote{We used the SRILM toolkit \cite{stolcke2002srilm} for training models and measuring perplexity.}.

Our examination shows that neither none of the parts of the Szeged Corpus contains as much abbreviated terms as clinical texts have. 
Likewise, sentences written by clinicians are significantly shorter than those of the Szeged Corpus. 
Neither the calculations above, nor the ratio of unknown words suggests using any of the  subcorpora for training. 
However, the perplexity scores contradict: sentences from the law domain share the most phrases with clinical notes, while news texts have the most similar grammatical structures. 

\begin{table}[H]
\centering
\caption{Evaluation of the tagger using the subcorpora as training data -- dev. set}
\label{tab:eval_subcorpora}
\begin{tabular}{ l r } 
\hline
Corpus & Morph. disambiguation accuracy \\
\hline
Szeged Corpus & \underline{94.73\%} \\
\hspace{0.2cm} Fiction & 92.01\% \\
\hspace{0.2cm} Compositions & 91.97\% \\
\hspace{0.2cm} Computer & 92.73\% \\
\hspace{0.2cm} Newspaper & \underline{93.29\%} \\
\hspace{0.2cm} Law & 92.17\% \\
\hspace{0.2cm} Short business news & 92.69\% \\
\hline
\end{tabular}
\end{table}


Since similarity measurements were not in accordance with each other, all sub-corpora were tested as training data for tagging clinical texts. 
(These experiments were performed using the previously enhanced tagging chain.)
The accuracy scores of taggers (cf. Table  \ref{tab:eval_subcorpora}) on the development set show that training on a subcorpus cannot improve the performance. 
Therefore, we decided to use the whole Szeged Corpus to train our system.

\subsection{Evaluation}

The improved chain (cf. Table \ref{tab:improvements}) was evaluated by investigating the part-of-speech tagging, lemmatization and the whole morphological annotation performance.

\begin{table}[H]
\centering
\caption{Evaluation of the improved tagger -- test set}
\label{tab:improvements}
\begin{tabular}{ l l r r r} 
\hline
ID & Method & PoS tagging & Lemmatization & Morph. disambig. \\
\hline
0 & Baseline system & 90.57\% & 93.54\% & 88.09\% \\
1 & 0 + Lexicon extension & 93.89\% & 96.24\% & 92.41\% \\
2 & 1 + Handling abbreviations & \underline{94.81\%} & \underline{97.60\%} & \underline{93.73\%} \\
%3 & 2 + Training data selection & 94.25\% & 97.36\% & 93.29\% \\
\hline
\end{tabular}
\end{table}

First of all, results show that the baseline method annotated almost 12\% of the tokens erroneously, while our enhancements raised the ceiling of the full morphological tagging accuracy to 93.73\%.
Therefore, we managed to eliminate almost half (47.36\%) of the errors. 
Next, precision scores also indicate that the error rate reduction is mainly due to the extended lexicon.
However, the better handling of abbreviations also increased the performance significantly (Wilcoxon test of paired samples, p < 0.05).
Therefore, our improvements yielded a system having satisfactory performance for morphologically parsing clinical texts.


This study revealed that abbreviations and \acrlong{oov} words cause the most of the errors for tagging Hungarian clinical texts.
We introduced numerous enhancements dealing with them, although not all of them were successful.
This could be due to the small amount of annotated data used inhibiting the better modeling of the domain.
%Further on, Hungarian is a less-resource language and as such we did not manage to find a better (sub)corpus for tagging clinical Hungarian than the whole Szeged Corpus. 


