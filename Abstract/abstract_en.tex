% ************************** Thesis Abstract *****************************
% Use `abstract' as an option in the document class to print only the titlepage and the abstract.

% összesen 350 szó (1 oldal)
\begin{abstract}

This thesis deals with text processing applications examining methods suitable for less-resourced and agglutinative languages.
In that way, we present accurate preprocessing algorithms for Hungarian.

The first part of this study deals with morphological tagging algorithms, which can compute both the morphosyntactic tags and lemmata of words accurately. 
A tool (called PurePos) was developed that was shown to produce precise annotations for Hungarian texts and also serve as a good base of rule-based domain adaptation scenarios. 
Besides, we present a methodology for combining tagger systems raising the overall accuracy of Hungarian annotation systems.

Next, an application of the presented tagger is described which aims speech transcripts. 
In doing so, the first morphological disambiguation tool for spoken Hungarian is introduced.
Following this, a method is described, which utilizes the adapted PurePos system, for estimating morphosyntactic complexity of Hungarian speech transcripts automatically.

The third part of the study deals with the preprocessing of electronic health records.
On the one hand, a hybrid algorithm is presented for segmenting clinical texts into words and sentences accurately.
On the other hand, domain-specific enhancements of  PurePos is described showing that the resulted tagger has satisfactory performance on noisy medical records.

Finally, we summarize the main results of this study by presenting our theses. 
Further on, numerous applications of the presented methods are introduced helping in research of Hungarian.

% téma (2): research quiestion

% 1. rész: morf. egyért. alogrimtusok...

% 2. rész: ezek alkalmazása gyermeknyelvi beszédátiratok összetettségének elemzésére

% 3. rész: klinikai szüvegek előfeldolgozása

% Összefoglalás
%  * adaptálható morf. egyért algoritm
%  * azok alkalmazását
%  * 




\end{abstract}
